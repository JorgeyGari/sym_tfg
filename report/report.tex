% !TeX program = xelatex
\documentclass[en]{uc3mthesisIEEE}
\usepackage[utf8]{inputenc} 

\usepackage{import}
\usepackage{enumitem}  % control item separation -> \begin{itemize}[nosep]
\usepackage{lipsum}  % dummy text
\usepackage{placeins}  % \FloatBarrier -> prevents figures and tables from passing that point
\usepackage{microtype}  % improves the spacing between words and letters
\usepackage{amsmath}  % math tools
\usepackage{listings, listings-rust}  % code listings
\renewcommand{\lstlistingname}{Listing}  % change caption name
\usepackage{appendix}  % appendices
\renewcommand{\appendixname}{Appendix}
\renewcommand{\appendixpagename}{Appendices}
\renewcommand{\appendixtocname}{Appendices}
\usepackage[nounderscore]{syntax}  % BNF grammars
\usepackage{tikz-uml}  % UML diagrams

\usepackage{mymacros}  % report-specific macros

% silence ht warnings
\usepackage{silence}
\WarningFilter{latex}{`h' float specifier changed to `ht'}


% REFERENCES
\addbibresource{references.bib}  % bibliography file
\import{}{glossary.tex}  % glossary file


%	DOCUMENT

% setup
\degree{Bachelor's Degree in Applied Mathematics and Computing}
\course{2023-2024}
\title{Development of a Symbolic Calculator from Scratch}
% \shorttitle{Short title}
\author{Jorge Lázaro Ruiz}
\tutors{Carlos Linares López}
\place{Leganés, Madrid, Spain}
\date{June 2024}

\begin{document}

  % COVER
  \makecover

  % % EPIGRAPH
  % \makeepigraph
  %   {Quote.}  % quote
  %   {Author}  % author
  %   {Placeholder quote}  % source


  % ABSTRACT
  \begin{abstract}
    Symbolic calculators are a type of software that can perform exact calculations with symbolic expressions, as opposed to numerical approximations. A complete study on the history of this relatively young research field is performed in order to extract conclusions and trends from previous computer algebra systems. The project aims to develop a symbolic calculator from scratch, without relying on external libraries or software, and written in the Rust programming language. The result is a program that can parse mathematical expressions, simplify them, and perform elementary operations like addition, subtraction, multiplication, and division. The program can also evaluate expressions for specific values of the variables, and can solve equations by finding the exact roots of linear and quadratic polynomials. The program is intended as a tool to help perform symbolic calculations in a simple and user-friendly way, and to provide exact solutions to mathematical problems in an easier, faster way than by hand. It is designed with a focus on simplicity and ease of use, and it is intended to be an experimental calculator that can be used to perform basic symbolic calculations effortlessly.
    \keywords{CAS, symbolic computation, computer algebra, calculator, elementary algebra, Rust}
  \end{abstract}


  % % ACKNOWLEDGEMENTS
  \begin{acknowledgements}
    First and foremost, thank you to my parents for their support and encouragement throughout my studies. It is because of them that I have been able to pursue my passion, even when things got tough. They have fought beside me each and every one of the battles I have had to overcome. In part, this thesis is also theirs.

    Thank you to my family who have kept me company in this journey, to those who have stayed with me until the end and those who couldn't make it. Special thanks to Mariano López, a brilliant mathematician who has shared his time and expertise with me during my studies. His mentoring has been invaluable.

    I would also like to thank my tutor, Professor Carlos Linares, for his advice during the development of the project. It is an honor to have worked with him, and I am grateful for the opportunity to learn from him.

    Thank you to the Charles III University of Madrid and the many professors I have been fortunate to learn from these past years. They have inspired me and helped me grow as a student, as a scientist, and as a person.

    Finally, thank you to my friends and to Marta, for their support, their understanding, and the memories we have created together. 
    Special thanks to my friend Luis Daniel Casais, who designed and generously shared the template used for this thesis.
  \end{acknowledgements}


  % TOC
  \tableofcontents
  \listoffigures
  \listoftables
  \lstlistoflistings


  % THESIS
  \begin{thesis}
    \includefrom{parts/}{introduction.tex}
    \includefrom{parts/}{state_of_the_art.tex}
    \includefrom{parts/}{design.tex}
    \includefrom{parts/}{implementation.tex}
    \includefrom{parts/}{planning.tex}
    \includefrom{parts/}{conclusions.tex}
  \end{thesis}

  % BIBLIOGRAPHY
  \cleardoublepage
  \label{bibliography}
  \printbibliography[heading=bibintoc]

  % GLOSSARY
  \cleardoublepage
  \label{glossary}
	\printglossaries
	% \printnoidxglossaries[type=\acronymtype]  % slower, but no need to do $ makeglossaries report


  % APPENDICES
  \begin{appendices}
    \includefrom{parts/}{appendix_a.tex}
    % \includefrom{parts/}{manual.tex}
  \end{appendices}

\end{document}