\chapter{Socio-economic Environment and Regulatory Framework}\label{chap:planning}

\section{Budget}\label{sec:budget}

Due to the nature of the project, the cost of development is mainly comprised of the time and tools it required. 

Table \ref{tab:human-resources} shows the estimated cost of the personnel, which is the main cost of the project.

The cost for the work hours of the student, who played the role of developer, is calculated based on the average salary of a software developer, €35 per hour, and the estimated time spent on the project, 330 hours, equivalent to 12 ECTS.

\begin{table}[H]
    \begin{tabular}{l r r r}
        \hline
        \textbf{Role} & \textbf{Cost per hour (€)} & \textbf{Hours} & \textbf{Total (€)} \\
        \hline
        Developer & 35 & 330 & 11,550 \\
        \hline
        \textbf{\textit{Total}} & & & \textbf{11,550.00} \\
        \hline
    \end{tabular}
    \caption{Estimated cost of human resources for the project.}
    \label{tab:human-resources}
\end{table}

Table \ref{tab:equipment} shows the cost of the equipment used for the project.

The specifics of the equipment used are as follows:

\begin{itemize}
    \item \textbf{Computer}: Lenovo ThinkPad T470s. The developer used his personal computer for the project. The current cost of the computer is estimated at €300.
    \item \textbf{Monitor}: ASUS ProArt Display. A second monitor was used for the project. The current cost of the monitor is estimated at €300.
    \item \textbf{Tablet}: Apple iPad Pro 2021. A tablet was used as the go-to device outside direct development for planning the project. The current cost of the tablet is estimated at €850.
    \item \textbf{Internet}: The cost of the internet connection is estimated at €30 per month. Since the project lasted for 5 months, the total cost is €150.
    \item \textbf{Software}: The developer used free software for the project, so there is no cost associated with it.
\end{itemize}

\begin{table}[H]
    \begin{tabular}{l r}
        \hline
        \textbf{Equipment} & \textbf{Cost (€)} \\
        \hline
        Computer & 300 \\
        Monitor & 300 \\
        Tablet & 850 \\
        Internet & 150 \\
        Software & 0 \\
        \hline
        \textbf{\textit{Total}} & \textbf{1,600.00} \\
        \hline
    \end{tabular}
    \caption{Estimated cost of equipment for the project.}
    \label{tab:equipment}
\end{table}

Table \ref{tab:total-cost} shows the total cost of the project.

\begin{table}[H]
    \begin{tabular}{l r}
        \hline
        \textbf{Item} & \textbf{Cost (€)} \\
        \hline
        Human Resources & 11,550 \\
        Equipment & 1,600 \\
        \hline
        \textbf{\textit{Total}} & \
        \textbf{13,150.00} \\
        \hline
    \end{tabular}
    \caption{Total cost of the project.}
    \label{tab:total-cost}
\end{table}

\section{Impact}\label{sec:impact}

Realistically, the project has a limited impact on the field, as it is a proof of concept and more of an experiment than a full-fledged product. However, the project has the potential to be expanded upon and developed further, and also adds to the body of knowledge in the field of computer algebra systems and symbolic computation.

A possible impact of the project is its use in educational settings. Symbotini can be used as a tool to teach students about elementary algebra.

Mathematics teachers can use Symbotini in the classroom in several ways:

\begin{itemize}
    \item \textbf{Student autonomy}: Students can use Symbotini to check their work and verify their answers. This can help them develop a sense of autonomy and self-reliance. It can also help them identify and correct their mistakes.
    \item \textbf{Less focus on computation}: By using Symbotini, teachers can focus less on computation and more on concepts and problem-solving strategies. More complex problems can be solved by the students with the help of Symbotini, eliminating the need for tedious and repetitive calculations and allowing the students to focus on the underlying concepts.
    \item \textbf{Improve understanding}: Symbotini can help students understand the concepts of algebra as they are encouraged to experiment with different expressions and see how they behave. This can help them develop a deeper understanding of the subject.
    \item \textbf{STEM education}: Symbotini can be used to introduce students to the field of computer algebra systems and symbolic computation. This can help spark an interest in STEM fields and encourage students to pursue further studies in these areas.
\end{itemize}

\section{Regulatory framework}\label{sec:regulatory-framework}

The project does not fall under any specific regulatory framework, as it does not involve any sensitive data or ethical considerations. There is no framework that regulates the development of computer algebra systems or symbolic calculators.

There exists a framework designed by Alasdair McAndrew of Victoria University for evaluating computer algebra systems in the context of mathematics education \parencite{mcandrewframework}. However, this framework is not a regulatory framework, but rather a set of guidelines for evaluating computer algebra systems in educational settings.

Because the project was written in Rust, it follows the style guidelines for the language as specified in the official style guide \parencite{rust-style-guide}.
