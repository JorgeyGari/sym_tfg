\chapter{Implementation}\label{chap:implementation}

\lstset{language=Rust, style=colouredRust, numbers=none}

\section{Design of the grammar}\label{sec:design-of-the-grammar}

Symbotini uses a custom-defined grammar to parse the input file in which the user specifies the desired calculations, expressions and assignments.

This grammar is a parsing expression grammar (PEG). It is defined and designed for the Pest parser generator \parencite{pest-book}, which is a Rust library for defining parsers. The grammar is defined in its own \verb|*.pest| file, which is then used by Pest to generate the parser code.

Symbotini's grammar supports implicit whitespace. This means that the parser automatically skips over whitespace characters in the input file, unless they are explicitly required by the grammar. For instance, inputting \verb|1+2| is equivalent to \verb|1 + 2|, or even \verb|1 +2| or \verb|1+ 2|.

The syntax for Pest parsers allows for two special kinds of rules: silent rules and atomic rules:

\begin{itemize}
    \item Silent rules, denoted by the underscore (\verb|_|) character, are rules that do not produce pairs or tokens. In this grammar, they are used for grouping other rules together or for the special case of the \verb|WHITESPACE| rule.
    \item Atomic rules, denoted by the at (\verb|@|) character, are rules preventing implicit whitespace. Additionally, any other rules called within an atomic rule are treated as silent rules.
\end{itemize}

Below is the grammar for Symbotini:

\begin{minipage}{\linewidth}
    \begin{lstlisting}[caption={The grammar for Symbotini}, label={lst:grammar}]
        expr = _{ assign | operation | polynomial | solve }
        
        operation = { "(" ~ polynomial ~ ")" ~ op ~ "(" ~ polynomial ~ ")" ~ (op ~ "(" ~ polynomial ~ ")")* }
        op        = _{ add | sub | mul | div }
        add       =  { "+" }
        sub       =  { "-" }
        mul       =  { "*" }
        div       =  { "/" }
    
        assign = { var ~ "=" ~ (operation | polynomial) }
        
        sign       =  { "+" | "-" }
        number     =  { ASCII_DIGIT+ ~ ("." ~ ASCII_DIGIT+)? }
        fraction   =  { number ~ "/" ~ number }
        var        = @{ var_name ~ ("^" ~ "(" ~ sign? ~ number ~ ("/" ~ number)? ~ ")")? }
        var_name   =  { ASCII_ALPHA }
        term       =  { sign? ~ (number | "(" ~ fraction ~ ")" | var)+ }
        polynomial =  { term ~ (WHITESPACE* ~ term)* }
        file       =  { SOI ~ (expr ~ NEWLINE?)* ~ EOI }
        WHITESPACE = _{ " " }
    
        solve = { "[" ~ polynomial ~ ("," ~ var_name)? ~ "]" }
    \end{lstlisting}
\end{minipage}

The first rule that will be matched is the \verb|file| rule. It is the entry point of the grammar and matches the entire input file. It consists of a sequence of expressions, each followed by an optional newline character. The \verb|SOI| and \verb|EOI| rules are special rules that match the start and end of the input, respectively.

The \verb|expr| rule is a silent rule that matches any of the following rules: \verb|assign|, \verb|operation|, \verb|polynomial| or \verb|solve|. This rule is used to group the different types of expressions that can be found in the input file, and all the operations the program can perform.

The purpose of the \verb|assign| rule is to match an assignment expression. It consists of a variable (\verb|var|) followed by an equal sign (\verb|=|) token and then either an \verb|operation| or a \verb|polynomial|.

The \verb|operation| rule matches one \verb|polynomial| followed by an operator (\verb|op|) and another \verb|polynomial|, and then optionally more operators and polynomials, making it possible to chain operations.

The \verb|op| rule, short for "operator", is a silent rule that matches any of the following rules: \verb|add|, \verb|sub|, \verb|mul| or \verb|div|. These rules are atomic and match the corresponding operator token.

The \verb|solve| rule matches a polynomial followed by an optional variable name enclosed in square brackets (\verb|[]|). The decision to use square brackets arises from the unambiguous nature of this syntax and PEGs in general, as it is not used elsewhere in the grammar. Using a keyword in the input like \verb|sol| would cause conflict, as the letters in the keyword could be mistaken by the parser for variables in cases like \verb|sol -x + 10|.

The \verb|polynomial| rule matches a sequence of \verb|term|s separated by optional whitespace.

A \verb|term| is defined as an optional \verb|sign| (either a \verb|+| token or a \verb|-| token) followed by a either a sequence of \verb|number|s, \verb|fraction|s (enclosed by parentheses to avoid mathematical ambiguity in cases like \verb|2/3x|, which could otherwise be interpreted as either $\frac{2}{3x}$ or $\frac{2}{3}x$) and variables (at least one of either). The decision to make the \verb|sign| optional is to allow for positive numbers to be written without the \verb|+| sign.

The \verb|var| rule matches a variable name followed by an optional exponentiation denoted by a caret symbol (\verb|^|). The exponentiation is enclosed in parentheses and consists of an optional \verb|sign|, a \verb|number| (optionally followed by a slash and another \verb|number|, for ratios). Exponentiation is optional in order to allow for variables to be written without an exponent (exponent $1$ is allowed to be implicit). The decision to make the \verb|var| rule atomic is to make sure the variable names are parsed correctly.

A \verb|var_name| (short for "variable name") is a single alphabetical ASCII character. This is a design decision to keep the grammar simple and to avoid conflicts with other rules. The user is restricted to alphabetical ASCII characters in order to allow the program to insert some special variables with non-alphabetical names like is done with the imaginary unit $\mathrm{i}$.

The \verb|number| rule matches a sequence of one or more digits followed by an optional decimal part, if the user wants to input a floating-point number. The decimal part is optional to prevent the user from having to write the decimal part if it is zero.

The \verb|fraction| rule matches a \verb|number| followed by a slash (\verb|/|) token and another \verb|number|. This rule is used to allow the user to input fractions in a more readable way.

The \verb|WHITESPACE| rule is a silent rule that matches a single space character. It is used to allow for implicit whitespace in the input file.

This grammar underwent several iterations and changes during the development of Symbotini. The final version of the grammar was chosen to be simple and easy to understand, while still being able to parse the desired input expressions. Many of the design decisions were made to make the input file as readable and natural as possible.

Using this grammar, Pest generates a parser that can analyze the input file and match each line with a rule. This allows Symbotini to understand the user's input, construct suitable structures and perform the desired calculations.

\section{Data structures}\label{sec:data-structures}

Symbotini uses several custom-defined data structures to represent the different elements of the input expressions and the results of the calculations. These data structures are defined as part of a Rust module which is imported by the main program.

The implementation of the methods mentioned in this section can be found in Appendix \ref{chap:appendix-a}.

\section{Class Diagram}

\begin{center}
\begin{tikzpicture}
\umlclass[y=2]{Variable}{
    name: String \\
    degree: Rational64
}{}

\umlclass[x=7, y=2]{Term}{
    coefficient: Rational64 \\
    variables: Vec<Variable>
}{
    max\_degree() : Rational64 \\
    sort\_vars() : void \\
    factor() : void \\
    invert() : void \\
    pow(q: Rational64) : Polynomial
}

\umlclass[x=0, y=-8, width=20em]{Polynomial}{
    terms: Vec<Term> \\
    degree: Rational64
}{
    degree() : Rational64 \\
    leading\_term() : Term \\
    evaluate(values: Vec<(String, Rational64)>) : void \\
    sort\_terms() : void \\
    as\_string() : String \\
    add\_like\_terms() : void \\
    simplify() : void \\
    make\_integer() : i64 \\
    factor() : (Term, Polynomial) \\
    first\_var() : Option<String> \\
    find\_sym\_coeff(var: \&str, degree: Rational64) : (Term, Term) \\
    roots(var: \&str) : Vec<Vec<PolyRatio{>}>
}

\umlclass[x=8, y=-8]{PolyRatio}{
    numerator: Polynomial \\
    denominator: Polynomial
}{
    simplify() : void \\
    as\_string() : String \\
    evaluate(values: Vec<(String, Rational64)>) : void
}

\umlassoc[mult1=1, mult2=*]{Polynomial}{Term}
\umlassoc[mult1=1, mult2=*]{Term}{Variable}
\umlassoc[mult1=1, mult2=1]{PolyRatio}{Polynomial}
\umlinherit[geometry=-|]{PolyRatio}{Term}
\end{tikzpicture}
\end{center}

\subsection{Rational64}\label{subsec:ratio64}

The type \verb|Rational64| is an alias for the \verb|Ratio<i64>| type from the \verb|num| crate. This type represents a rational number as a ratio of two integers (numerator and denominator). The \verb|num| crate provides a generic implementation of arithmetic operations for rational numbers.

Although this type is from the crate \verb|num| and was not created for this project, its use in Symbotini is essential for the correct representation of fractions in the input expressions.

\subsection{Variable}\label{subsec:variable}

\begin{minipage}{\linewidth}
    \begin{lstlisting}[caption={The \texttt{Variable} struct}, label={lst:variable}]
        pub struct Variable {
            pub name: String,
            pub degree: Rational64,
        }
    \end{lstlisting}
\end{minipage}

The \verb|Variable| struct represents a variable in a polynomial expression. It consists of a name (a \verb|String|) and a degree (a \verb|Rational64|). The name is a string containing the variable's name, which is a single alphabetical ASCII character. The degree is a rational number representing the variable's exponent.

A custom definition of equality for the \verb|Variable| struct was implemented. Two variables are considered equal if their names are equal and their degrees are equal. This is used to compare variables in the polynomial simplification process.

\subsection{Term}\label{subsec:term}

\begin{minipage}{\linewidth}
    \begin{lstlisting}[caption={The \texttt{Term} struct}, label={lst:term}]
        pub struct Term {
            pub coefficient: Rational64,
            pub variables: Vec<Variable>,
        }    
    \end{lstlisting}
\end{minipage}

The \verb|Term| struct represents a term in a polynomial expression. It consists of a rational coefficient (a \verb|Rational64|) and a vector of variables (\verb|Vec<Variable>|), defined in the custom \texttt{Variable} struct.

There are several methods implemented for the \verb|Term| struct:

\begin{itemize}
    \item \verb|sort_vars()| is used to sort the variables in the term by lexico-graphical order of their name, i.e. the term $4yx$ would be rewritten as $4xy$. It is used to simplify the process of comparing terms.
    \item \verb|max_degree()| is used to find the maximum degree of the variables in the term. It iterates over the variables in the term and returns the variable with the highest degree. If the term has no variables, it returns $0$.
    \item \verb|factor()| is used to factor the term by combining like variables (variables with the same name). It iterates over the variables in the term and combines their coefficients if their names are equal. This method is used in the polynomial simplification process.
    \item \verb|invert()| is used to invert the term by multiplying the coefficient and every variable's degree by $-1$.
    \item \verb|pow()| is used to raise the term to a power. This method is selective and only raises the term to a power if the result isn't "uglier" than the original term. For instance, raising $x^2$ to the power of $2$ would result in $x^4$, and raising $25$ to the power of $0.5$ would result in $5$ (inside of a \verb|Polynomial|), but raising $13$ to the power of $0.5$ will just return $13^{\frac{1}{2}}$ in a \verb|Polynomial| struct, rather than a ratio approximating the value of $\sqrt{13}$.
\end{itemize}

% \multilinecomment{

% }

In addition to these methods, the \verb|Term| struct has a custom implementation of equality. Two terms are considered equal if their coefficients are equal and their variable vectors, after being sorted, are equal. This is used to compare terms in the polynomial simplification process.

% \multilinecomment{

% }

Multiplication and division of terms are also custom implementations in the \verb|Term| struct. For multiplication, the method multiplies the coefficients and concatenates the variable vectors. For division, the method upgrades both \verb|Term|s into \verb|Polynomial|s and divides them. The result is a \verb|Polynomial| with a single \verb|Term|: the quotient.

% \multilinecomment{

% }

\subsection{Polynomial}\label{subsec:polynomial}

\begin{minipage}{\linewidth}
    \begin{lstlisting}[caption={The \texttt{Polynomial} struct}, label={lst:polynomial}]
        pub struct Polynomial {
            pub terms: Vec<Term>,
            pub degree: Rational64,
        }
    \end{lstlisting}
\end{minipage}


The \verb|Polynomial| struct represents a polynomial expression. It consists of a vector of terms (\verb|Vec<Term>|) and a degree (\verb|Rational64|). The degree is a rational number representing an exponent to which the polynomial is raised, allowing for the representation of expressions like $\sqrt{2x + y}$.

The \verb|Polynomial| struct has several methods implemented:

\begin{itemize}
    \item \verb|degree()| is used to find the degree of the polynomial. It iterates over the terms in the polynomial and returns the term with the highest degree, using the \verb|max_degree()| method from the \texttt{Term} struct.
    \item \verb|leading_term()| is used to find the leading term of the polynomial. It iterates over the terms in the polynomial and returns the term with the highest degree according to the \verb|max_degree()| method from the \texttt{Term} struct.
    \item \verb|evaluate()| is used to evaluate the polynomial given a vector of tuples containing variable names and their values. It iterates over the terms in the polynomial and evaluates each term, substituting the variable values into the term when applicable.
    \item \verb|sort_terms()| is used to sort the terms in the polynomial by the degree of their leading variable and then by alphabetical order. It is used to simplify the process of comparing polynomials and to convert the polynomial to what is considered its canonical form.
    \item \verb|as_string()| is used to convert the polynomial to a string like \verb|2x^(2) + 3y + 4| for improved readability when printing. It makes several decisions to make the output more readable, like omitting the coefficient if it is $1$ or $-1$ and specifying the degree if it is different from $1$.
    \item \verb|add_like_terms()| is used to add the terms with the same variables in the polynomial. It iterates over the terms in the polynomial and combines the terms adding up their coefficients if they have the same vector of variables once it has been sorted.
    \item \verb|simplify()| is the general simplification method for a polynomial. It first checks if it can get rid of the polynomial's exponent (\verb|degree|) either using the \verb|pow()| method from the \texttt{Term} struct or multiplying the polynomial by itself as many times as \verb|degree| specifies. Then it sorts each term's variables, factors them and adds like terms. Finally, it removes terms with a coefficient of $0$ and, if the polynomial is empty after this (meaning the expression was equal to $0$), it adds a null term.
    \item \verb|make_integer()| is a utility method for \verb|factor()|ing polynomials. It finds the smallest scalar such that all of the polynomial's terms have integer coefficients and performs the operation. It works by finding the least common multiple of the denominators of the coefficients of the terms and multiplying all the coefficients by this scalar. 
    \item \verb|factor()| tries to find the greatest common divisor term that can be factored out from the polynomial. It computes the greatest common divisor of the terms's coefficients iterating over them. Each of the variable names of the first term is compared with the variable names of the other terms to filter the collection until only common variables among all of the terms remain. If there is a common variable, then the lowest degree that variable is raised to in the polynomial is found. A term is constructed based on the results of this procedure and is factored out of the polynomial. This whole process is aided by the \verb|make_integer()| method to make calculations easier, a transformation which is undone at the end.
    \item \verb|first_var()| returns the name of the first variable in the polynomial.
    \item \verb|find_sym_coeff()| is used to find the (symbolic) coefficient of a specific variable and degree in the polynomial, i.e. the coefficient of $x^2$ in $2ax^2 + 3ax + 4$ is $2a$. This is useful for finding a polynomial's roots when applying, for instance, the quadratic formula.
    \item \verb|roots()| defines the algorithm for finding the roots of a polynomial. It is composed of two parts: one for linear equations and one for quadratic equations.
    \begin{itemize}
        \item For a linear equation of the form $ax + b = 0$, the function finds the coefficient $a$ (dividing the degree-$1$ term by the variable $x$) and identifies the remaining of the polynomial as $b$. After that it returns the root $-b/a$.
        \item For a quadratic equation of the form $ax^2 + bx + c = 0$, the roots are found using the quadratic formula, $x = \frac{-b \pm \sqrt{b^2 - 4ac}}{2a}$. The coefficients, $a$, $b$ and $c$, are found analogously to the linear case, and the discriminant is calculated as $\Delta = b^2 - 4ac$. If $\Delta < 0$, the function must handle complex roots. This is done by multiplying $\Delta$ by $-1$ and by the square of the imaginary unit $\mathrm{i}$, represented as the special \verb|Variable| struct with unicode character U+2148 (DOUBLE-STRUCK ITALIC SMALL I) as its \verb|name| and with \verb|degree| $2$, obtaining an equivalent expression to the original discriminant. The square root of this expression is then calculated, and the roots are found as $x = \frac{-b \pm \sqrt{\Delta}}{2a}$. These are returned as a vector of \verb|Term|s because of the special way complex roots are displayed to the user in the main program.  % I can't get the unicode character to display.
    \end{itemize}
\end{itemize}

% \multilinecomment{

% }

The \verb|Polynomial| struct also has a custom implementation of equality. Two polynomials are considered equal if, once simplified, their vector of terms are equal.

% \multilinecomment{

% }

Operations for \verb|Polynomial|s are also defined as custom functions:

\begin{itemize}
    \item Adding two \verb|Polynomial|s results in a \verb|Polynomial| structure whose vector of \verb|Term|s contains the terms of both summands. The sum is then simplified.
    \item Subtracting a \verb|Polynomial| from another is done by negating the second \verb|Polynomial| (iterating over its terms and multiplying their coefficients by $-1$) and adding it to the first \verb|Polynomial|.
    \item Multiplication of two \verb|Polynomial|s is done by iterating through each term of the original polynomials. For each pair of terms, a new term is createed with the coefficient being the product of the original coefficients and the variables being the combination of the original variables. This new term is added to a result vector. After processing all terms, the product polynomial is created with these terms, and is then simplified and returned.
    \item The division of two \verb|Polynomial|s is based on the Euclidean division algorithm:
    \begin{enumerate}
        \item Initialize the quotient \( Q(x) = 0 \) and the remainder \( R(x) = P(x) \).
        \item While the degree of \( R(x) \) is greater than or equal to the degree of \( D(x) \):
        \begin{enumerate}
            \item Find the leading term (term with the highest degree) of the current remainder \( R(x) \).
            \[
            \text{Let } \text{LT}(R(x)) = a_n x^n
            \]
            \item Divide the leading term of \( R(x) \) by the leading term of \( D(x) \) to obtain a new term \( T(x) \).
            \[
            T(x) = \frac{\text{LT}(R(x))}{\text{LT}(D(x))} = \frac{a_n x^n}{b_m x^m} = \left( \frac{a_n}{b_m} \right) x^{n-m}
            \]
            \item Add \( T(x) \) to the quotient \( Q(x) \).
            \[
            Q(x) = Q(x) + T(x)
            \]
            \item Subtract \( T(x) \cdot D(x) \) from the remainder \( R(x) \).
            \[
            R(x) = R(x) - T(x) \cdot D(x)
            \]
        \end{enumerate}
        \item Continue this process until the degree of \( R(x) \) is less than the degree of \( D(x) \).
    \end{enumerate}
\end{itemize}

% \multilinecomment{

% }

\subsection{PolyRatio}\label{subsec:polyratio}

\begin{minipage}{\linewidth}
    \begin{lstlisting}[caption={The \texttt{PolyRatio} struct}, label={lst:polyratio}]
        pub struct PolyRatio {
            pub numerator: Polynomial,
            pub denominator: Polynomial,
        }
    \end{lstlisting}
\end{minipage}

\verb|PolyRatio| represents a rational algebraic expression, with a numerator and a denominator, both of type \verb|Polynomial|. This function is the most complex and powerful in the module.

The \verb|PolyRatio| struct has several methods implemented:

\begin{itemize}
    \item The \verb|simplify()| function is responsible for simplifying a fraction of polynomials.
    
    \begin{enumerate}
        \item It first simplifies the numerator and denominator separately, converts coefficients to integers, and adjusts for negative exponents in both. It then multiplies the numerator and denominator by the accumulated terms.
        \item Next, it \verb|factor()|s both the numerator and denominator: it identifies common variables in the factored terms, calculates the minimum degree and the greatest common divisor (GCD) of the coefficients and uses the GCD to cancel terms in the numerator and denominator.
        \item Finally, it undoes the coefficient scaling. If the numerator and denominator are equal, it simplifies the fraction to $1$.
    \end{enumerate}

    \item The method \verb|as_string()| converts the \verb|PolyRatio| to a string like \texttt{(2x + 3) / (4x - 5)} for improved readability when printing. It calls the \verb|as_string()| method from the \verb|Polynomial| struct, but it includes parentheses around the numerator and denominator. This means that all simplification in output done by \verb|Polynomial|'s \verb|as_string()| method is conveniently inherited by \verb|PolyRatio|'s \verb|as_string()| method. Some additional logic is included to handle the case where the denominator is $1$---in which case only the numerator is shown---and an error message in case the user tries to input a division by zero.
    \item The \verb|evaluate()| method is used to evaluate the \verb|PolyRatio| given a vector of tuples containing variable names and their values. It \verb|evaluate()|s the numerator and denominator \verb|Polynomial|s separately.

\end{itemize}

% \multilinecomment{

% }

The four elementary arithmetic operations are defined for \verb|PolyRatio| in the following way:
\begin{itemize}
    \item Addition is implemented by multiplying the denominators and adding the numerators, then simplifying the result.
    \item Subtraction is implemented by multiplying the denominators and subtracting the numerators, then simplifying the result.
    \item Multiplication is implemented by multiplying the numerators and denominators separately, then simplifying the result.
    \item Division is implemented by multiplying the numerator by the denominator of the divisor and the denominator by the numerator of the divisor (cross-multiplication), then simplifying the result. 
\end{itemize}

% \multilinecomment{

% }

A set of compatibility functions are also implemented for \verb|PolyRatio| to make it easier to work with \verb|Polynomial|s:

\begin{itemize}
    \item The \verb|from()| function converts a \verb|Polynomial| to a \verb|PolyRatio| by setting the denominator to $1$.
    \item The four elementary arithmetic operations are implemented between \verb|Polynomial| and \verb|PolyRatio|. These operations are implemented by upgrading the \verb|Polynomial| using \verb|from()| and performing the operation as defined for \verb|PolyRatio|. These are actually methods of the \verb|Polynomial| struct.
\end{itemize}

% \multilinecomment{

% }

\section{Main program}\label{sec:main-program}

The main program is responsible for parsing, evaluating and performing operations on the symbolic expressions expressed by the custom syntax defined in the grammar. Using the Pest parser, the program reads the input file and performs the operations.

The implementation of the functions mentioned in this section can be found in Appendix \ref{chap:appendix-a}.

\subsection{Helper functions}\label{subsec:helper-functions}

The main program has several helper functions that are used to parse the input file:

\begin{itemize}
    \item The \verb|variable_from_string()| function converts a string (e.g.: \verb|x^(2)|) to a \verb|Variable| struct. It splits the string by the caret symbol (\verb|^|) if present, then converts the first part to a string and the second part (which can be a ratio, a floating-point number or an integer) to a \verb|Rational64|, and returns a \verb|Variable| struct.
    \item The \verb|parse_polynomial()| function parses a polynomial from the result of matching the \verb|polynomial| rule in the Pest parser (e.g.: \verb|x^(2) + 3y - 1|). It iterates through the parts of the parsed expression, identifying terms and factors within terms, according to the inner rules of the definition of a polynomial according to the grammar. The terms are accumulated and used to construct a \verb|Polynomial| struct.
    \item The \verb|parse_assignment()| function parses an assignment from the result of matching the \verb|assignment| rule in the Pest parser (e.g.: \verb|x = 2|). It extracts the variable name and the value, and stores them in a tuple, later to be pushed to the vector holding the variable-value tuples.
    \item The \verb|parse_operation()| function parses an operation from the result of matching the \verb|operation| rule in the Pest parser (e.g.: \verb|(4x) * (y + 6) / (2)|) into a \verb|PolyRatio| struct. It parses the first polynomial in the operation sequence, and then iterates through the rest of the operation sequence, parsing the next polynomial, matching the operation with its corresponding rule and performing the calculation using the previous result.
\end{itemize}

% \multilinecomment{

% }

\subsection{Main function}\label{subsec:main-function}

The main function orchestrates the overall flow of the program by reading input, parsing it, and performing various operations based on the parsed data.

First, it reads the input file into a string. If the file cannot be read, the programm will panic and terminate with an error message indicating the issue.

If the file is read successfully, the program will parse the input using the Pest parser that was generated based on the custom grammar. If the input cannot be parsed, the program will panic with an error message indicating the issue.

After the two previous steps have been successfuly executed, we can initialize storage for the vector \verb|var_values|, where we will store the variable-value tuples with all the custom definition the user inputs.

The program proceeds with the parsing of each of the lines in the input file. When a line is read, it will be skipped if it is empty. If not, the input line will be printed to let the user know what operation is being performed.

Each of the lines is matched to one of the rules defined in the grammar as possible \verb|expr|(ession) types. This includes a \verb|polynomial|, an \verb|assignment|, an \verb|operation|, or an equation to \verb|solve|.

Here is an overview of the behavior when matching each of the rules:
\begin{itemize}
    \item \textbf{Assignment}:
    \begin{itemize}
        \item Call the \verb|parse_assignment()| function to extract the variable name and value.
        \item Add the variable-value tuple to the \verb|var_values| vector.
        \item Print the assignment to the user.
    \end{itemize}
    \item \textbf{Polynomial}:
    \begin{itemize}
        \item Call the \verb|parse_polynomial()| function to parse the polynomial in the line.
        \item Evaluate the polynomial using the stored variable values.
        \item Print the evaluation and simplification of polynomial.
    \end{itemize}
    \item \textbf{Operation}:
    \begin{itemize}
        \item Call the \verb|parse_operation()| function to parse the operation in the line.
        \item Evaluate the resulting \verb|PolyRatio| using the stored variable values.
        \item Print the evaluation and simplification of the result of the operation.
    \end{itemize}
    \item \textbf{Solve}:
    \begin{itemize}
        \item Call the \verb|parse_polynomial()| function to parse the equation in the line. It is assumed that the right-hand side of the equation is always $0$.
        \item Identify which variable is being solved for. The user may have specified it after the equation (e.g.: \texttt{[3x - 1, x]}). If not, then the first variable in the equation is assumed to be the one being solved for.
        \begin{itemize}
            \item In case no variable is found in the equation, the program will panic and terminate with an error message. 
        \end{itemize}
        \item Solve the equation for the variable by calling the \verb|solve()| method of the \verb|Polynomial| struct.
        \item Print the solution to the user, handling the case where there are multiple solutions.
    \end{itemize}
\end{itemize}

An additional rule exists for handling end of input (EOI). Any other unreachable cases that should not occur if the grammar is properly followed in the input file will cause the program to panic and terminate with an error message.

Refer to the appendix for the full implementation of the main program, divided into the parts mentioned in this section.
