\chapter{Implementation}\label{chap:implementation}

\lstset{language=Rust, style=colouredRust, numbers=none}

\section{Design of the grammar}\label{sec:design-of-the-grammar}

Symbotini uses a custom-defined grammar to parse the input file in which the user specifies the desired calculations, expressions and assignments.

This grammar is a parsing expression grammar (PEG). It is defined and designed for the Pest parser generator \parencite{pest-book}, which is a Rust library for defining parsers. The grammar is defined in its own \verb|*.pest| file, which is then used by Pest to generate the parser code.

Symbotini's grammar supports implicit whitespace. This means that the parser automatically skips over whitespace characters in the input file, unless they are explicitly required by the grammar. For instance, inputting \verb|1+2| is equivalent to \verb|1 + 2|, or even \verb|1 +2| or \verb|1+ 2|.

The syntax for Pest parsers allows for two special kinds of rules: silent rules and atomic rules:

\begin{itemize}
    \item Silent rules, denoted by the underscore (\verb|_|) character, are rules that do not produce pairs or tokens. In this grammar, they are used for grouping other rules together or for the special case of the \verb|WHITESPACE| rule.
    \item Atomic rules, denoted by the at (\verb|@|) character, are rules preventing implicit whitespace. Additionally, any other rules called within an atomic rule are treated as silent rules.
\end{itemize}

Below is the grammar for Symbotini:

\begin{lstlisting}[caption={The grammar for Symbotini}, label={lst:grammar}]
    expr = _{ assign | operation | polynomial | solve }
    
    operation = { "(" ~ polynomial ~ ")" ~ op ~ "(" ~ polynomial ~ ")" ~ (op ~ "(" ~ polynomial ~ ")")* }
    op        = _{ add | sub | mul | div }
    add       =  { "+" }
    sub       =  { "-" }
    mul       =  { "*" }
    div       =  { "/" }

    assign = { var ~ "=" ~ (operation | polynomial) }
    
    sign       =  { "+" | "-" }
    number     =  { ASCII_DIGIT+ ~ ("." ~ ASCII_DIGIT+)? }
    fraction   =  { number ~ "/" ~ number }
    var        = @{ var_name ~ ("^" ~ "(" ~ sign? ~ number ~ ("/" ~ number)? ~ ")")? }
    var_name   =  { ASCII_ALPHA }
    term       =  { sign? ~ (number | "(" ~ fraction ~ ")" | var)+ }
    polynomial =  { term ~ (WHITESPACE* ~ term)* }
    file       =  { SOI ~ (expr ~ NEWLINE?)* ~ EOI }
    WHITESPACE = _{ " " }

    solve = { "[" ~ polynomial ~ ("," ~ var_name)? ~ "]" }
\end{lstlisting}

Let us explain the grammar in more detail.

The first rule that will be matched is the \verb|file| rule. It is the entry point of the grammar and matches the entire input file. It consists of a sequence of expressions, each followed by an optional newline character. The \verb|SOI| and \verb|EOI| rules are special rules that match the start and end of the input, respectively.

The \verb|expr| rule is a silent rule that matches any of the following rules: \verb|assign|, \verb|operation|, \verb|polynomial| or \verb|solve|. This rule is used to group the different types of expressions that can be found in the input file, and all the operations the program can perform.

The purpose of the \verb|assign| rule is to match an assignment expression. It consists of a variable (\verb|var|) followed by an equal sign (\verb|=|) token and then either an \verb|operation| or a \verb|polynomial|.

The \verb|operation| rule matches one \verb|polynomial| followed by an operator (\verb|op|) and another \verb|polynomial|, and then optionally more operators and polynomials, making it possible to chain operations.

The \verb|op| rule, short for "operator", is a silent rule that matches any of the following rules: \verb|add|, \verb|sub|, \verb|mul| or \verb|div|. These rules are atomic and match the corresponding operator token.

The \verb|solve| rule matches a polynomial followed by an optional variable name enclosed in square brackets (\verb|[]|). The decision to use square brackets arises from the unambiguous nature of this syntax and PEGs in general, as it is not used elsewhere in the grammar. Using a keyword in the input like \verb|sol| would cause conflict, as the letters in the keyword could be mistaken by the parser for variables in cases like \verb|sol -x + 10|.

The \verb|polynomial| rule matches a sequence of \verb|term|s separated by optional whitespace.

A \verb|term| is defined as an optional \verb|sign| (either a \verb|+| token or a \verb|-| token) followed by a either a sequence of \verb|number|s, \verb|fraction|s (enclosed by parentheses to avoid mathematical ambiguity in cases like \verb|2/3x|, which could otherwise be interpreted as either $\frac{2}{3x}$ or $\frac{2}{3}x$) and variables (at least one of either). The decision to make the \verb|sign| optional is to allow for positive numbers to be written without the \verb|+| sign.

The \verb|var| rule matches a variable name followed by an optional exponentiation denoted by a caret symbol (\verb|^|). The exponentiation is enclosed in parentheses and consists of an optional \verb|sign|, a \verb|number| (optionally followed by a slash and another \verb|number|, for ratios). Exponentiation is optional in order to allow for variables to be written without an exponent (exponent $1$ is allowed to be implicit). The decision to make the \verb|var| rule atomic is to make sure the variable names are parsed correctly.

A \verb|var_name| (short for "variable name") is a single alphabetical ASCII character. This is a design decision to keep the grammar simple and to avoid conflicts with other rules. The user is restricted to alphabetical ASCII characters in order to allow the program to insert some special variables with non-alphabetical names like is done with the imaginary unit $\mathrm{i}$.

The \verb|number| rule matches a sequence of one or more digits followed by an optional decimal part, if the user wants to input a floating-point number. The decimal part is optional to prevent the user from having to write the decimal part if it is zero.

The \verb|fraction| rule matches a \verb|number| followed by a slash (\verb|/|) token and another \verb|number|. This rule is used to allow the user to input fractions in a more readable way.

The \verb|WHITESPACE| rule is a silent rule that matches a single space character. It is used to allow for implicit whitespace in the input file.

This grammar underwent several iterations and changes during the development of Symbotini. The final version of the grammar was chosen to be simple and easy to understand, while still being able to parse the desired input expressions. Many of the design decisions were made to make the input file as readable and natural as possible.

Using this grammar, Pest generates a parser that can analyze the input file and match each line with a rule. This allows Symbotini to understand the user's input, construct suitable structures and perform the desired calculations.

\section{Data structures}\label{sec:data-structures}

Symbotini uses several custom-defined data structures to represent the different elements of the input expressions and the results of the calculations. These data structures are defined as part of a Rust module which is imported by the main program.

\subsection{Rational64}\label{subsec:ratio64}

The type \verb|Rational64| is an alias for the \verb|Ratio<i64>| type from the \verb|num| crate. This type represents a rational number as a ratio of two integers (numerator and denominator). The \verb|num| crate provides a generic implementation of arithmetic operations for rational numbers.

Although this type is from the crate \verb|num| and was not created for this project, its use in Symbotini is essential for the correct representation of fractions in the input expressions.

\subsection{Variable}\label{subsec:variable}

\begin{lstlisting}[caption={The \texttt{Variable} struct}, label={lst:variable}]
    pub struct Variable {
        pub name: String,
        pub degree: Rational64,
    }
\end{lstlisting}

The \verb|Variable| struct represents a variable in a polynomial expression. It consists of a name (a \verb|String|) and a degree (a \verb|Rational64|). The name is a string containing the variable's name, which is a single alphabetical ASCII character. The degree is a rational number representing the variable's exponent.

A custom definition of equality for the \verb|Variable| struct was implemented. Two variables are considered equal if their names are equal and their degrees are equal. This is used to compare variables in the polynomial simplification process.

\begin{lstlisting}[caption={The implementation of \texttt{PartialEq} for the \texttt{Variable} struct}, label={lst:variable-eq}]
    impl PartialEq for Variable {
        fn eq(&self, other: &Self) -> bool {
            self.name == other.name && self.degree == other.degree
        }
    }
\end{lstlisting}

\subsection{Term}\label{subsec:term}

\begin{lstlisting}[caption={The \texttt{Term} struct}, label={lst:term}]
    pub struct Term {
        pub coefficient: Rational64,
        pub variables: Vec<Variable>,
    }    
\end{lstlisting}

The \verb|Term| struct represents a term in a polynomial expression. It consists of a rational coefficient (a \verb|Rational64|) and a vector of variables (\verb|Vec<Variable>|), defined in the custom \texttt{Variable} struct.

There are several methods implemented for the \verb|Term| struct:

\begin{itemize}
    \item \verb|sort_vars()| is used to sort the variables in the term by lexico-graphical order of their name, i.e. the term $4yx$ would be rewritten as $4xy$. It is used to simplify the process of comparing terms.
    \item \verb|max_degree()| is used to find the maximum degree of the variables in the term. It iterates over the variables in the term and returns the variable with the highest degree. If the term has no variables, it returns $0$.
    \item \verb|factor()| is used to factor the term by combining like variables (variables with the same name). It iterates over the variables in the term and combines their coefficients if their names are equal. This method is used in the polynomial simplification process.
    \item \verb|invert()| is used to invert the term by multiplying every variable's degree by $-1$ and multiplying the coefficient by $-1$.
    \item \verb|pow()| is used to raise the term to a power. This method is selective and only raises the term to a power if the result isn't "uglier" than the original term. For instance, raising $x^2$ to the power of $2$ would result in $x^4$, and raising $25$ to the power of $0.5$ would result in $5$ (both inside a \verb|Polynomial|), but raising $13$ to the power of $0.5$ will just return $13^{\frac{1}{2}}$ in a \verb|Polynomial| struct, rather than a ratio approximating the value of $\sqrt{13}$.
\end{itemize}

% \multilinecomment{
    \begin{minipage}{\linewidth}
        \begin{lstlisting}[caption={The implementation of the \texttt{sort\_vars()} method for the \texttt{Term} struct}, label={lst:term-sort-vars}]
            /// Sorts the variables in the term in ascending order based on their names.
            pub fn sort_vars(&mut self) {
                self.variables.sort_by(|a, b| a.name.cmp(&b.name));
            }
        \end{lstlisting}
    \end{minipage}

    \begin{minipage}{\linewidth}
        \begin{lstlisting}[caption={The implementation of the \texttt{max\_degree()} method for the \texttt{Term} struct}, label={lst:term-max-degree}]
            /// Find max degree of the variables in the term.
            pub fn max_degree(&self) -> Rational64 {
                self.variables
                    .iter()
                    .map(|v| v.degree)
                    .max()
                    .unwrap_or(0.into())
            }
        \end{lstlisting}
    \end{minipage}

    \begin{lstlisting}[caption={The implementation of the \texttt{factor()} method for the \texttt{Term} struct}, label={lst:term-factor}]
        /// Factors the term by combining like variables.
        pub fn factor(&mut self) {
            let mut new_vars: Vec<Variable> = Vec::new();
            for var1 in &self.variables {
                let mut found = false;
                for var2 in &mut new_vars {
                    if var1.name == var2.name {
                        var2.degree += var1.degree;
                        found = true;
                        break;
                    }
                }
                if !found {
                    new_vars.push(var1.clone());
                }
            }
            new_vars.retain(|v| v.degree != 0.into());
            self.variables = new_vars;
        }
    \end{lstlisting}

    \begin{minipage}{\linewidth}
        \begin{lstlisting}[caption={The implementation of the \texttt{invert()} method for the \texttt{Term} struct}, label={lst:term-invert}]
            /// Inverts the term.
            pub fn invert(&mut self) {
                self.coefficient = Rational64::new(*self.coefficient.denom(), *self.coefficient.numer());
                for var in &mut self.variables {
                    var.degree *= -1;
                }
            }
        \end{lstlisting}
    \end{minipage}

    \begin{lstlisting}[caption={The implementation of the \texttt{pow()} method for the \texttt{Term} struct}, label={lst:term-pow}]
        /// Returns a polynomial containing the term to the power of q.
        pub fn pow(&self, q: Rational64) -> Polynomial {
            let mut new_vars: Vec<Variable> = Vec::new();
            for var in &self.variables {
                new_vars.push(Variable {
                    name: var.name.clone(),
                    degree: var.degree * q.clone(),
                });
            }
            let mut ratio_coef =
                Rational64::from_f64(self.coefficient.to_f64().unwrap().powf(q.to_f64().unwrap()))
                    .unwrap();
            if self.coefficient.denom() == &1 && ratio_coef.denom() != &1 {
                // Don't convert expressions like sqrt(13) to a ratio
                return Polynomial {
                    terms: vec![self.clone()],
                    degree: q,
                };
            }
            Polynomial {
                terms: vec![Term {
                    coefficient: ratio_coef,
                    variables: new_vars,
                }],
                degree: 1.into(),
            }
        }
    \end{lstlisting}
% }

In addition to these methods, the \verb|Term| struct has a custom implementation of equality. Two terms are considered equal if their coefficients are equal and their variable vectors, after being sorted, are equal. This is used to compare terms in the polynomial simplification process.

% \multilinecomment{
    \begin{lstlisting}[caption={The implementation of \texttt{PartialEq} for the \texttt{Term} struct}, label={lst:term-eq}]
        impl PartialEq for Term {
            fn eq(&self, other: &Self) -> bool {
                self.coefficient == other.coefficient
                    && {
                        let mut self_vars = self.variables.clone();
                        let mut other_vars = other.variables.clone();
                        self_vars.sort();
                        other_vars.sort();
                        self_vars == other_vars
                    }
            }
        }
    \end{lstlisting}
% }

Multiplication and division of terms are also custom implementations in the \verb|Term| struct. For multiplication, the method multiplies the coefficients and concatenates the variable vectors. For division, the method upgrades both \verb|Term|s into \verb|Polynomial|s and divides them. The result is a \verb|Polynomial| with a single \verb|Term|: the quotient.

% \multilinecomment{
    \begin{lstlisting}[caption={The implementation of the multiplication operation for the \texttt{Term} struct}, label={lst:term-mul}]
        impl Mul for Term {
            type Output = Term;
            fn mul(self, other: Self) -> Term {
                let mut result = Vec::new();
                let mut new_vars = self.variables.clone();
                new_vars.extend(other.variables.clone());
                let mut new_term = Term {
                    coefficient: self.coefficient * other.coefficient,
                    variables: new_vars,
                };
                new_term.sort_vars();
                new_term.factor();
                result.push(new_term);
                let mut product = Polynomial {
                    terms: result,
                    degree: 1.into(),
                };
                product.simplify();
                product.terms[0].clone()
            }
        }
    \end{lstlisting}

    \begin{lstlisting}[caption={The implementation of the division operation for the \texttt{Term} struct}, label={lst:term-div}]
        impl Div for Term {
            type Output = Polynomial;
            fn div(self, other: Self) -> Polynomial {
                let dividend = Polynomial {
                    terms: vec![self],
                    degree: 1.into(),
                };
                let divisor = Polynomial {
                    terms: vec![other],
                    degree: 1.into(),
                };
                let mut result = Vec::new();
                for term1 in &dividend.terms {
                    for term2 in &divisor.terms {
                        let mut new_vars = term2.variables.clone();
                        for var in &mut new_vars {
                            var.degree *= -1;
                        }
                        new_vars.extend(term1.variables.clone());
                        let mut new_term = Term {
                            coefficient: term1.coefficient / term2.coefficient,
                            variables: new_vars,
                        };
                        new_term.sort_vars();
                        new_term.factor();
                        result.push(new_term);
                    }
                }
                let mut quotient = Polynomial {
                    terms: result,
                    degree: 1.into(),
                };
                quotient.simplify();
                quotient
            }
        }
    \end{lstlisting}
% }

\subsection{Polynomial}\label{subsec:polynomial}

\begin{lstlisting}[caption={The \texttt{Polynomial} struct}, label={lst:polynomial}]
    pub struct Polynomial {
        pub terms: Vec<Term>,
        pub degree: Rational64,
    }
\end{lstlisting}


The \verb|Polynomial| struct represents a polynomial expression. It consists of a vector of terms (\verb|Vec<Term>|) and a degree (\verb|Rational64|). The degree is a rational number representing an exponent to which the polynomial is raised, allowing for the representation of expressions like $\sqrt{2x + y}$.

The \verb|Polynomial| struct has several methods implemented:

\begin{itemize}
    \item \verb|degree()| is used to find the degree of the polynomial. It iterates over the terms in the polynomial and returns the term with the highest degree, using the \verb|max_degree()| method from the \texttt{Term} struct.
    \item \verb|leading_term()| is used to find the leading term of the polynomial. It iterates over the terms in the polynomial and returns the term with the highest degree according to the \verb|max_degree()| method from the \texttt{Term} struct.
    \item \verb|evaluate()| is used to evaluate the polynomial given a vector of tuples containing variable names and their values. It iterates over the terms in the polynomial and evaluates each term, substituting the variable values into the term when applicable.
    \item \verb|sort_terms()| is used to sort the terms in the polynomial by the degree of their leading variable and then by alphabetical order. It is used to simplify the process of comparing polynomials and to convert the polynomial to what is considered its canonical form.
    \item \verb|as_string()| is used to convert the polynomial to a string like \verb|2x^(2) + 3y + 4| for improved readability when printing. It makes several decisions to make the output more readable, like omitting the coefficient if it is $1$ or $-1$ and specifying the degree if it is different from $1$.
    \item \verb|add_like_terms()| is used to add the terms with the same variables in the polynomial. It iterates over the terms in the polynomial and combines the terms adding up their coefficients if they have the same vector of variables once it has been sorted.
    \item \verb|simplify()| is the general simplification method for a polynomial. It first checks if it can get rid of the polynomial's exponent (\verb|degree|) either using the \verb|pow()| method from the \texttt{Term} struct or multiplying the polynomial by itself as many times as \verb|degree| specifies. Then it sorts each term's variables, factors them and adds like terms. Finally, it removes terms with a coefficient of $0$ and, if the polynomial is empty after this (meaning the expression was equal to $0$), it adds a null term.
    \item \verb|make_integer()| is a utility method for \verb|factor()|ing polynomials. It finds the smallest scalar such that all of the polynomial's terms have integer coefficients and performs the operation. It works by finding the least common multiple of the denominators of the coefficients of the terms and multiplying all the coefficients by this scalar. 
    \item \verb|factor()| tries to find the greatest common divisor term that can be factored out from the polynomial. It computes the greatest common divisor of the terms's coefficients iterating over them. Each of the variable names of the first term is compared with the variable names of the other terms to filter the collection until only common variables among all of the terms remain. If there is a common variable, then the lowest degree that variable is raised to in the polynomial is found. A term is constructed based on the results of this procedure and is factored out of the polynomial. This whole process is aided by the \verb|make_integer()| method to make calculations easier, a transformation which is undone at the end.
    \item \verb|first_var()| returns the name of the first variable in the polynomial.
    \item \verb|find_sym_coeff()| is used to find the (symbolic) coefficient of a specific variable and degree in the polynomial, i.e. the coefficient of $x^2$ in $2ax^2 + 3ax + 4$ is $2a$. This is useful for finding a polynomial's roots when applying, for instance, the quadratic formula.
    \item \verb|roots()| defines the algorithm for finding the roots of a polynomial. It is composed of two parts: one for linear equations and one for quadratic equations.
    \begin{itemize}
        \item For a linear equation of the form $ax + b = 0$, the function finds the coefficient $a$ (dividing the degree-$1$ term by the variable $x$) and identifies the remaining of the polynomial as $b$. After that it returns the root $-b/a$.
        \item For a quadratic equation of the form $ax^2 + bx + c = 0$, the roots are found using the quadratic formula, $x = \frac{-b \pm \sqrt{b^2 - 4ac}}{2a}$. The coefficients, $a$, $b$ and $c$, are found analogously to the linear case, and the discriminant is calculated as $\Delta = b^2 - 4ac$. If $\Delta < 0$, the function must handle complex roots. This is done by multiplying $\Delta$ by $-1$ and by the square of the imaginary unit $\mathrm{i}$, represented as the special \verb|Variable| struct with unicode character U+2148 (DOUBLE-STRUCK ITALIC SMALL I) as its \verb|name| and with \verb|degree| $2$, obtaining an equivalent expression to the original discriminant. The square root of this expression is then calculated, and the roots are found as $x = \frac{-b \pm \sqrt{\Delta}}{2a}$. These are returned as a vector of \verb|Term|s because of the special way complex roots are displayed to the user in the main program.  % I can't get the unicode character to display.
    \end{itemize}
\end{itemize}

% \multilinecomment{
    \begin{lstlisting}[caption={The implementation of the \texttt{degree()} method for the \texttt{Polynomial} struct}, label={lst:polynomial-degree}]
        /// Return the degree of the polynomial.
        pub fn degree(&self) -> Rational64 {
            self.terms
                .iter()
                .map(|t| t.max_degree())
                .max()
                .unwrap_or(0.into())
        }
    \end{lstlisting}

    \begin{lstlisting}[caption={The implementation of the \texttt{leading\_term()} method for the \texttt{Polynomial} struct}, label={lst:polynomial-leading-term}]
        /// Return the leading term of the polynomial.
        pub fn leading_term(&self) -> Term {
            let mut max_degree = Rational64::new(0, 1);
            let mut leading_term = Term {
                coefficient: Rational64::new(0, 1),
                variables: Vec::new(),
            };
            for term in &self.terms {
                let degree = term.max_degree();
                if degree >= max_degree {
                    max_degree = degree;
                    leading_term = term.clone();
                }
            }
            leading_term
        }
    \end{lstlisting}

    \begin{lstlisting}[caption={The implementation of the \texttt{evaluate()} method for the \texttt{Polynomial} struct}, label={lst:polynomial-evaluate}]
        /// Evaluate the polynomial at a given value for the variables.
        pub fn evaluate(&mut self, values: &Vec<(String, Rational64)>) {
            let mut result = Polynomial {
                terms: Vec::new(),
                degree: 1.into(),
            };
            for term in &self.terms {
                let mut new_term = term.clone();
                for var in &mut new_term.variables {
                    if let Some(val) = values.iter().find(|(name, _)| name == &var.name) {
                        let value = *val.1.clone().numer() as f64 / *val.1.denom() as f64;
                        let expon = *var.degree.numer() as f64 / *var.degree.denom() as f64;
                        new_term.coefficient =
                            new_term.coefficient * Rational64::from_f64(value.powf(expon)).unwrap();
                        var.degree = 0.into(); // Set the degree of the variable to 0, essentially removing it from the term
                    }
                }
                result.terms.push(new_term);
            }
            *self = result;
            self.simplify();
        }
    \end{lstlisting}

    \begin{lstlisting}[caption={The implementation of the \texttt{sort\_terms()} method for the \texttt{Polynomial} struct}, label={lst:polynomial-sort-terms}]
        /// Sorts the terms in the polynomial in descending order based on the max degree of the variables in each term, then by alphabetical order.
        pub fn sort_terms(&mut self) -> () {
            self.terms.sort_by(|a, b| {
                let max_degree_cmp = b
                    .variables
                    .iter()
                    .map(|v| v.degree)
                    .max()
                    .unwrap_or(0.into())
                    .cmp(
                        &a.variables
                            .iter()
                            .map(|v| v.degree)
                            .max()
                            .unwrap_or(0.into()),
                    );
                if max_degree_cmp != Ordering::Equal {
                    return max_degree_cmp;
                }
                a.variables.cmp(&b.variables)
            });
        }
    \end{lstlisting}

    \begin{lstlisting}[caption={The implementation of the \texttt{as\_string()} method for the \texttt{Polynomial} struct}, label={lst:polynomial-as-string}]
        /// Converts the polynomial to a string in a pretty format.
        pub fn as_string(&self) -> String {
            let mut result = String::new();
            for (i, term) in self.terms.iter().enumerate() {
                if term.coefficient == Rational64::new(0, 1) && self.terms.len() > 1 {
                    continue;
                }
                if i != 0 && term.coefficient > Rational64::new(0, 1) {
                    result.push_str("+");
                }
                if term.variables.is_empty() || term.coefficient != Rational64::new(1, 1) {
                    if term.coefficient == Rational64::new(-1, 1) && !term.variables.is_empty() {
                        result.push_str("-");
                    } else {
                        result.push_str(&term.coefficient.to_string());
                    }
                }
                for variable in &term.variables {
                    result.push_str(&variable.name);
                    if variable.degree != 1.into() {
                        result.push_str(&format!("^({})", variable.degree));
                    }
                }
            }
            if self.degree != 1.into() {
                result = format!("({})^({})", result, self.degree);
            }
            result
        }
    \end{lstlisting}

    \begin{lstlisting}[caption={The implementation of the \texttt{add\_like\_terms()} method for the \texttt{Polynomial} struct}, label={lst:polynomial-add-like-terms}]
        /// Adds like terms in the polynomial.
        pub fn add_like_terms(&mut self) -> () {
            let mut new_terms: Vec<Term> = Vec::new();

            for term in &self.terms {
                let coeff: Rational64 = term.coefficient.clone();
                let mut found = false;

                for term1 in &mut new_terms {
                    if term1.variables == term.variables {
                        term1.coefficient += term.coefficient;
                        found = true;
                        break;
                    }
                }

                if !found {
                    new_terms.push(Term {
                        coefficient: coeff,
                        variables: term.variables.clone(),
                    });
                }
            }

            self.terms = new_terms;
        }
    \end{lstlisting}

    \begin{lstlisting}[caption={The implementation of the \texttt{simplify()} method for the \texttt{Polynomial} struct}, label={lst:polynomial-simplify}]
        /// Simplifies the polynomial by sorting the terms, sorting the variables in each term, factoring each term, and adding like terms.
        pub fn simplify(&mut self) -> () {
            let d: Option<f64> = self.degree.to_f64();
            // println!("Degree: {:?}", d);
            if d.is_some() {
                if self.terms.len() == 1 {
                    let exp = Rational64::from_f64(d.unwrap());
                    let powered = self.terms[0].pow(exp.unwrap());
                    self.terms = powered.terms;
                    self.degree = powered.degree;
                } else if d.unwrap().fract() == 0.0 && d.unwrap() >= 2.0 {
                    for _i in 1..d.unwrap() as i64 {
                        *self = self.clone() * self.clone();
                    }
                    self.degree = 1.into();
                }
            }
            // println!("Simplifying 1: {}", self.as_string());

            for term in &mut self.terms {
                term.sort_vars();
            }

            for term in &mut self.terms {
                term.factor();
            }

            self.add_like_terms();

            // Filter to remove terms with coefficient 0
            self.terms
                .retain(|term| term.coefficient != Rational64::new(0, 1));

            // Add a term with coefficient 0 if all terms were removed
            if self.terms.is_empty() {
                self.terms.push(Term {
                    coefficient: Rational64::new(0, 1),
                    variables: vec![],
                });
            }

            self.sort_terms();
        }
    \end{lstlisting}

    \begin{lstlisting}[caption={The implementation of the \texttt{make\_integer()} method for the \texttt{Polynomial} struct}, label={lst:polynomial-make-integer}]
        /// Multiplies the polynomial by the smallest scalar such that all coefficients are integers. Returns the scalar.
        pub fn make_integer(&mut self) -> i64 {
            // Get the lcm of the denominators of the coefficients
            let mut lcm = 1;
            for term in &self.terms {
                let denom = term.coefficient.denom();
                lcm = num_integer::lcm(lcm as i64, *denom);
            }
            // Multiply each coefficient by the lcm
            for term in &mut self.terms {
                term.coefficient *= Rational64::new(lcm, 1);
            }
            lcm
        }
    \end{lstlisting}

    \begin{lstlisting}[caption={The implementation of the \texttt{factor()} method for the \texttt{Polynomial} struct}, label={lst:polynomial-factor}]
        /// Finds the greatest common divisor of the coefficients of the terms in a single-variable polynomial with integer coefficients. Returns the gcd and the polynomial with the gcd factored out.
        pub fn factor(&mut self) -> (Term, Polynomial) {
            if self.degree != 1.into() {
                return (
                    Term {
                        coefficient: Rational64::new(1, 1),
                        variables: vec![],
                    },
                    self.clone(),
                ); // Only works for degree 1 polynomials
            }
            let mut factored_out = Term {
                coefficient: Rational64::new(1, 1),
                variables: vec![],
            };
            let mut factored = self.clone();

            // Find the gcd of the coefficients
            let mut gcd = self.terms[0].coefficient.numer().abs();
            for term in &self.terms {
                gcd = num_integer::gcd(gcd, term.coefficient.numer().abs());
            }
            factored_out.coefficient = Rational64::new(gcd, 1);

            // Check the name of the variable that appears in all terms
            let mut seen_vars = vec![];
            for var in &self.terms[0].variables {
                seen_vars.push(var.name.clone());
            }
            for term in &self.terms {
                let mut curr_vars = vec![];
                for var in &term.variables {
                    curr_vars.push(var.name.clone());
                }
                seen_vars = seen_vars
                    .iter()
                    .map(|x| x.clone())
                    .filter(|x| curr_vars.contains(x))
                    .collect();

                if seen_vars.len() == 0 {
                    if gcd != 0 {
                        for term in &mut factored.terms {
                            term.coefficient *= Rational64::new(1, gcd);
                        }
                    }
                    return (factored_out, factored);
                }
            }
            let var_name = seen_vars[0].clone(); // Always the first element

            // Make the coefficients integers
            let mut p = self.clone();
            let adjust = p.make_integer();
            p.simplify();

            // Find the gcd of the coefficients
            let mut gcd = p.terms[0].coefficient.numer().abs();
            for term in &p.terms {
                gcd = num_integer::gcd(gcd, term.coefficient.numer().abs());
            }

            // Find the smallest power of the variable that appears in all terms
            let mut min_degree = p.terms[0].variables[0].degree;
            for term in &p.terms {
                for var in &term.variables {
                    if var.degree < min_degree && var.degree > 0.into() {
                        min_degree = var.degree;
                    }
                }
            }

            // Factor out the gcd
            factored_out.coefficient = Rational64::new(gcd, 1);
            factored_out.variables.push(Variable {
                name: var_name,
                degree: min_degree,
            });

            let mut inv: Term = factored_out.clone();
            inv.invert();
            factored = factored
                * Polynomial {
                    terms: vec![inv],
                    degree: 1.into(),
                };

            // Undo the scaling of the coefficients
            for term in &mut factored.terms {
                term.coefficient *= Rational64::new(1, adjust);
            }
            factored_out.coefficient *= Rational64::new(1, adjust);

            factored.simplify();

            return (factored_out, factored);
        }
    \end{lstlisting}

    \begin{lstlisting}[caption={The implementation of the \texttt{first\_var()} method for the \texttt{Polynomial} struct}, label={lst:polynomial-first-var}]
        /// Returns the name of the first variable in the polynomial.
        pub fn first_var(&self) -> Option<String> {
            if self.terms.len() == 0 {
                panic!("Polynomial has no terms!");
            } else if self.terms[0].variables.len() == 0 {
                return None;
            } else {
                return Some(self.terms[0].variables[0].name.clone());
            }
        }
    \end{lstlisting}

    \begin{lstlisting}[caption={The implementation of the \texttt{find\_sym\_coeff()} method for the \texttt{Polynomial} struct}, label={lst:polynomial-find-sym-coeff}]
        /// Find symbolic coefficient by degree.
        pub fn find_sym_coeff(&self, var: &str, degree: Rational64) -> (Term, Term) {
                let zero = Term {
                    coefficient: Rational64::new(0, 1),
                    variables: vec![],
                    };
                    let term = self
                    .terms
                    .iter()
                    .find(|t| {
                        t.variables
                        .iter()
                        .any(|v| v.name == var && v.degree == degree)
                        })
                        .unwrap_or_else(|| &zero)
                        .clone();
                        let sym_coeff = Term {
                            coefficient: term.coefficient.clone(),
                            variables: term
                        .variables
                        .iter()
                        .filter(|v| v.name != var)
                        .cloned()
                        .collect(),
                };
                return (term, sym_coeff);
                }
    \end{lstlisting}

    \begin{lstlisting}[caption={The implementation of the \texttt{roots()} method for the \texttt{Polynomial} struct}, label={lst:polynomial-roots}]
        /// Finds the roots (numerical or symbolic) of the polynomial.
        pub fn roots(&self, var: &str) -> Vec<Vec<PolyRatio>> {
            let mut result = vec![Vec::new()];
            let mut self_copy = self.clone();
            self_copy.simplify();

            // Find out the degree of the polynomial, but only taking into account the variable var
            let degree = self_copy
                .terms
                .iter()
                .map(|t| {
                    t.variables
                        .iter()
                        .find(|v| v.name == var)
                        .map(|v| v.degree)
                        .unwrap_or(0.into())
                })
                .max()
                .unwrap_or(0.into());

            match degree {
                d if d == 1.into() => {
                    let (a_term, a) = self_copy.find_sym_coeff(var, 1.into());
                    let b = self.clone()
                        - Polynomial {
                            terms: vec![a_term],
                            degree: 1.into(),
                        };

                    let minus_b = PolyRatio::from(b)
                        * PolyRatio::from(Polynomial {
                            terms: vec![Term {
                                coefficient: Rational64::new(-1, 1),
                                variables: vec![],
                            }],
                            degree: 1.into(),
                        });
                    let root = minus_b
                        / PolyRatio::from(Polynomial {
                            terms: vec![a],
                            degree: 1.into(),
                        });
                    result.push(vec![root]);
                }
                d if d == 2.into() => {
                    // Find the term with x^2 by filtering the terms with the variable x and degree 2
                    let (a_term, a) = self.find_sym_coeff(var, 2.into());
                    let (b_term, b) = self.find_sym_coeff(var, 1.into());
                    let c: Polynomial = self.clone()
                        - Polynomial {
                            terms: vec![a_term.clone(), b_term.clone()],
                            degree: 1.into(),
                        };
                    let minus_b = PolyRatio::from(Polynomial {
                        terms: vec![Term {
                            coefficient: Rational64::new(-1, 1),
                            variables: vec![],
                        }],
                        degree: 1.into(),
                    }) * PolyRatio::from(Polynomial {
                        terms: vec![b.clone()],
                        degree: 1.into(),
                    });
                    let mut b_squared = PolyRatio::from(Polynomial {
                        terms: vec![b.clone()],
                        degree: 2.into(),
                    });
                    b_squared.simplify();
                    let four_ac = PolyRatio::from(Polynomial {
                        terms: vec![Term {
                            coefficient: Rational64::new(4, 1),
                            variables: vec![],
                        }],
                        degree: 1.into(),
                    }) * PolyRatio::from(Polynomial {
                        terms: vec![a.clone()],
                        degree: 1.into(),
                    }) * c.clone();
                    let discriminant = b_squared.clone() - four_ac.clone();
                    let mut sqrt_discriminant = discriminant.clone();
                    if (discriminant.numerator.terms[0].coefficient < 0.into())
                        ^ (discriminant.denominator.terms[0].coefficient < 0.into())
                    {
                        sqrt_discriminant = discriminant.clone()
                            * PolyRatio::from(Polynomial {
                                terms: vec![Term {
                                    coefficient: Rational64::new(-1, 1),
                                    variables: vec![],
                                }],
                                degree: 1.into(),
                            })
                            * PolyRatio::from(Polynomial {
                                terms: vec![Term {
                                    coefficient: Rational64::new(1, 1),
                                    variables: vec![Variable {
                                        name: "\u{2148}".to_string(),
                                        degree: Rational64::new(2, 1),
                                    }],
                                }],
                                degree: 1.into(),
                            });
                        println!("(\u{2148} is the imaginary unit)");
                    }
                    sqrt_discriminant.numerator.degree = Rational64::new(1, 2);
                    sqrt_discriminant.denominator.degree = Rational64::new(1, 2);
                    sqrt_discriminant.simplify();
                    let two_a = PolyRatio::from(Polynomial {
                        terms: vec![Term {
                            coefficient: Rational64::new(2, 1),
                            variables: vec![],
                        }],
                        degree: 1.into(),
                    }) * PolyRatio::from(Polynomial {
                        terms: vec![a.clone()],
                        degree: 1.into(),
                    });
                    if sqrt_discriminant.numerator.degree != Rational64::new(1, 1) {
                        let root1 = vec![
                            minus_b.clone() / two_a.clone(),
                            sqrt_discriminant.clone() / two_a.clone(),
                        ];
                        let root2 = vec![
                            minus_b.clone() / two_a.clone(),
                            sqrt_discriminant.clone()
                                / (PolyRatio::from(Polynomial {
                                    terms: vec![Term {
                                        coefficient: Rational64::new(-1, 1),
                                        variables: vec![],
                                    }],
                                    degree: 1.into(),
                                }) * two_a.clone()),
                        ];
                        result.push(root1);
                        result.push(root2);
                    } else {
                        let root1 = (minus_b.clone() + sqrt_discriminant.clone()) / two_a.clone();
                        let root2 = (minus_b - sqrt_discriminant) / two_a;
                        result.push(vec![root1]);
                        result.push(vec![root2]);
                    }
                }
                _ => {
                    panic!("Higher degree polynomials not supported!");
                }
            }
            return result;
        }
    \end{lstlisting}
% }

The \verb|Polynomial| struct also has a custom implementation of equality. Two polynomials are considered equal if, once simplified, their vector of terms are equal.

% \multilinecomment{
    \begin{lstlisting}[caption={The implementation of \texttt{PartialEq} for the \texttt{Polynomial} struct}, label={lst:polynomial-eq}]
    impl PartialEq for Polynomial {
        fn eq(&self, other: &Self) -> bool {
            let mut self_copy = self.clone();
            self_copy.simplify();
            let mut other_copy = other.clone();
            other_copy.simplify();
            self_copy.terms == other_copy.terms
        }
    }
    \end{lstlisting}
% }

Operations for \verb|Polynomial|s are also defined as custom functions:

\begin{itemize}
    \item Adding two \verb|Polynomial|s results in a \verb|Polynomial| structure whose vector of \verb|Term|s contains the terms of both summands. The sum is then simplified.
    \item Subtracting a \verb|Polynomial| from another is done by negating the second \verb|Polynomial| (iterating over its terms and multiplying their coefficients by $-1$) and adding it to the first \verb|Polynomial|.
    \item Multiplication of two \verb|Polynomial|s is done by iterating through each term of the original polynomials. For each pair of terms, a new term is createed with the coefficient being the product of the original coefficients and the variables being the combination of the original variables. This new term is added to a result vector. After processing all terms, the product polynomial is created with these terms, and is then simplified and returned.
    \item The division of two \verb|Polynomial|s is based on the Euclidean division algorithm:
    \begin{enumerate}
        \item Initialize the quotient \( Q(x) = 0 \) and the remainder \( R(x) = P(x) \).
        \item While the degree of \( R(x) \) is greater than or equal to the degree of \( D(x) \):
        \begin{enumerate}
            \item Find the leading term (term with the highest degree) of the current remainder \( R(x) \).
            \[
            \text{Let } \text{LT}(R(x)) = a_n x^n
            \]
            \item Divide the leading term of \( R(x) \) by the leading term of \( D(x) \) to obtain a new term \( T(x) \).
            \[
            T(x) = \frac{\text{LT}(R(x))}{\text{LT}(D(x))} = \frac{a_n x^n}{b_m x^m} = \left( \frac{a_n}{b_m} \right) x^{n-m}
            \]
            \item Add \( T(x) \) to the quotient \( Q(x) \).
            \[
            Q(x) = Q(x) + T(x)
            \]
            \item Subtract \( T(x) \cdot D(x) \) from the remainder \( R(x) \).
            \[
            R(x) = R(x) - T(x) \cdot D(x)
            \]
        \end{enumerate}
        \item Continue this process until the degree of \( R(x) \) is less than the degree of \( D(x) \).
    \end{enumerate}
\end{itemize}

% \multilinecomment{
    \begin{lstlisting}[caption={The implementation of the addition operation for the \texttt{Polynomial} struct}, label={lst:polynomial-add}]
        impl Add for Polynomial {
            type Output = Self;
        
            fn add(self, other: Self) -> Self {
                let mut result = self.terms.clone();
                result.extend(other.terms);
                let mut sum = Polynomial {
                    terms: result,
                    degree: 1.into(),
                };
                sum.simplify();
                sum
            }
        }        
    \end{lstlisting}

    \begin{lstlisting}[caption={The implementation of the subtraction operation for the \texttt{Polynomial} struct}, label={lst:polynomial-sub}]
        impl Sub for Polynomial {
            type Output = Self;
        
            fn sub(self, mut other: Self) -> Self {
                for term in &mut other.terms {
                    term.coefficient *= -1;
                }
        
                self.add(other)
            }
        }        
    \end{lstlisting}

    \begin{lstlisting}[caption={The implementation of the multiplication operation for the \texttt{Polynomial} struct}, label={lst:polynomial-mul}]
        impl Mul for Polynomial {
            type Output = Self;
        
            fn mul(self, other: Self) -> Self {
                let mut result = Vec::new();
                if other
                    == (Polynomial {
                        terms: vec![Term {
                            coefficient: Rational64::new(1, 1),
                            variables: vec![],
                        }],
                        degree: 1.into(),
                    })
                {
                    return self;
                }
                for term1 in &self.terms {
                    for term2 in &other.terms {
                        let mut new_vars = term1.variables.clone();
                        new_vars.extend(term2.variables.clone());
                        let mut new_term = Term {
                            coefficient: term1.coefficient * term2.coefficient,
                            variables: new_vars,
                        };
                        new_term.sort_vars();
                        new_term.factor();
                        result.push(new_term);
                    }
                }
                let mut product = Polynomial {
                    terms: result,
                    degree: 1.into(),
                };
                product.simplify();
                product
            }
        }
    \end{lstlisting}
    
    \begin{lstlisting}[caption={The implementation of the division operation for the \texttt{Polynomial} struct}, label={lst:polynomial-div}]
        impl Div for Polynomial {
            type Output = PolyRatio;
            fn div(self, other: Self) -> PolyRatio {
                let mut dividend = self.clone();
                dividend.simplify();
        
                if dividend.terms.len() == 0 {
                    return PolyRatio::from(Polynomial {
                        terms: vec![Term {
                            coefficient: Rational64::new(0, 1),
                            variables: vec![],
                        }],
                        degree: 1.into(),
                    });
                }
        
                let mut divisor = other.clone();
                divisor.simplify();
        
                let mut quotient = Polynomial {
                    terms: vec![],
                    degree: 1.into(),
                };
        
                let mut remainder = dividend.clone();
        
                let zero_poly = Polynomial {
                    terms: vec![Term {
                        coefficient: Rational64::new(0, 1),
                        variables: vec![],
                    }],
                    degree: 1.into(),
                };
        
                if remainder.degree() < divisor.degree() {
                    return PolyRatio {
                        numerator: remainder,
                        denominator: divisor,
                    };
                }
        
                while remainder != zero_poly
                    && remainder.terms.len() != 0
                    && remainder.degree() >= divisor.degree()
                {
                    let t = remainder.leading_term() / divisor.leading_term();
                    quotient = quotient + t.clone();
                    remainder = remainder - (divisor.clone() * t.clone());
                    remainder.simplify();
                }
        
                quotient.simplify();
                let ratio = PolyRatio::from(quotient);
                ratio
            }
        }
    \end{lstlisting}        
% }

\subsection{PolyRatio}\label{subsec:polyratio}

\begin{lstlisting}[caption={The \texttt{PolyRatio} struct}, label={lst:polyratio}]
    pub struct PolyRatio {
        pub numerator: Polynomial,
        pub denominator: Polynomial,
    }
\end{lstlisting}

\verb|PolyRatio| represents a rational algebraic expression, with a numerator and a denominator, both of type \verb|Polynomial|. This function is the most complex and powerful in the module.

The \verb|PolyRatio| struct has several methods implemented:

\begin{itemize}
    \item The \verb|simplify()| function is responsible for simplifying a fraction of polynomials.
    
    \begin{enumerate}
        \item It first simplifies the numerator and denominator separately, converts coefficients to integers, and adjusts for negative exponents in both. It then multiplies the numerator and denominator by the accumulated terms.
        \item Next, it \verb|factor()|s both the numerator and denominator: it identifies common variables in the factored terms, calculates the minimum degree and the greatest common divisor (GCD) of the coefficients and uses the GCD to cancel terms in the numerator and denominator.
        \item Finally, it undoes the coefficient scaling. If the numerator and denominator are equal, it simplifies the fraction to $1$.
    \end{enumerate}

    \item The method \verb|as_string()| converts the \verb|PolyRatio| to a string like \texttt{(2x + 3) / (4x - 5)} for improved readability when printing. It calls the \verb|as_string()| method from the \verb|Polynomial| struct, but it includes parentheses around the numerator and denominator. This means that all simplification in output done by \verb|Polynomial|'s \verb|as_string()| method is conveniently inherited by \verb|PolyRatio|'s \verb|as_string()| method. Some additional logic is included to handle the case where the denominator is $1$---in which case only the numerator is shown---and an error message in case the user tries to input a division by zero.
    \item The \verb|evaluate()| method is used to evaluate the \verb|PolyRatio| given a vector of tuples containing variable names and their values. It \verb|evaluate()|s the numerator and denominator \verb|Polynomial|s separately.

\end{itemize}

% \multilinecomment{
    \begin{lstlisting}[caption={The implementation of the \texttt{simplify()} method for the \texttt{PolyRatio} struct}, label={lst:polyratio-simplify}]
        pub fn simplify(&mut self) {
            self.numerator.simplify();
            self.denominator.simplify();

            // Make the coefficients integers
            let mut n = self.numerator.clone();
            let mut d = self.denominator.clone();
            let adjust_n = n.make_integer();
            let adjust_d = d.make_integer();
    
            // Find the smallest negative exponent of each variable in the denominator
            let mut vars_to_move: Vec<Variable> = vec![];
            for term in &d.terms {
                for var in &term.variables {
                    if var.degree < 0.into() {
                        vars_to_move.push(var.clone());
                    }
                }
            }
            for var in &mut vars_to_move {
                var.degree *= -1;
            }
    
            // Multiply the numerator and denominator by the accumulated terms
            n = n * Polynomial {
                terms: vec![Term {
                    coefficient: Rational64::new(1, 1),
                    variables: vars_to_move.clone(),
                }],
                degree: 1.into(),
            };
            d = d * Polynomial {
                terms: vec![Term {
                    coefficient: Rational64::new(1, 1),
                    variables: vars_to_move.clone(),
                }],
                degree: 1.into(),
            };
    
            // Find the smallest negative exponent of each variable in the denominator
            let mut vars_to_move: Vec<Variable> = vec![];
            for term in &n.terms {
                for var in &term.variables {
                    if var.degree < 0.into() {
                        vars_to_move.push(var.clone());
                    }
                }
            }
            for var in &mut vars_to_move {
                var.degree *= -1;
            }
    
            // Multiply the numerator and denominator by the accumulated terms
            n = n * Polynomial {
                terms: vec![Term {
                    coefficient: Rational64::new(1, 1),
                    variables: vars_to_move.clone(),
                }],
                degree: 1.into(),
            };
            d = d * Polynomial {
                terms: vec![Term {
                    coefficient: Rational64::new(1, 1),
                    variables: vars_to_move.clone(),
                }],
                degree: 1.into(),
            };
    
            // Factor out as much as possible from the numerator and denominator
            let (t1, mut n) = n.factor();
            let (t2, mut d) = d.factor();
    
            // We are going to divide the numerator and denominator, these are the values by default
            let mut var_name = "".to_string();
            let mut min_degree = 0.into();
    
            // If the factored out terms share a variable
            if t1.variables.len() != 0 && t2.variables.len() != 0 {
                if t1.variables[0].name == t2.variables[0].name {
                    var_name = t1.variables[0].name.clone();
                    min_degree = t1.variables[0].degree.min(t2.variables[0].degree);
                }
            }
    
            let gcd_term = Term {
                // The term that will be canceled out in the numerator and denominator
                coefficient: Rational64::from_integer(num_integer::gcd(
                    t1.coefficient.numer().abs(),
                    t2.coefficient.numer().abs(),
                )),
                variables: if var_name != "" {
                    // If the terms share a variable
                    vec![Variable {
                        name: var_name,
                        degree: min_degree,
                    }]
                } else {
                    vec![]
                },
            };

            n = n * Polynomial {
                terms: vec![t1.clone()],
                degree: 1.into(),
            };

            d = d * Polynomial {
                terms: vec![t2],
                degree: 1.into(),
            };
    
            // Cancel out the gcd from the numerator and denominator
            let mut inv = gcd_term.clone();
            inv.invert();
            n = n * Polynomial {
                terms: vec![inv.clone()],
                degree: 1.into(),
            };
            d = d * Polynomial {
                terms: vec![inv],
                degree: 1.into(),
            };
    
            // Undo the scaling of the coefficients
            for term in &mut n.terms {
                term.coefficient *= Rational64::new(1, adjust_n);
            }
            for term in &mut d.terms {
                term.coefficient *= Rational64::new(1, adjust_d);
            }
    
            self.numerator = n;
            self.denominator = d;
    
            self.numerator.simplify();
            self.denominator.simplify();
    
            if self.denominator == self.numerator {
                self.numerator = Polynomial {
                    terms: vec![Term {
                        coefficient: Rational64::new(1, 1),
                        variables: vec![],
                    }],
                    degree: 1.into(),
                };
                self.denominator = Polynomial {
                    terms: vec![Term {
                        coefficient: Rational64::new(1, 1),
                        variables: vec![],
                    }],
                    degree: 1.into(),
                };
            }
        }
    \end{lstlisting}

    \begin{lstlisting}[caption={The implementation of the \texttt{as\_string()} method for the \texttt{PolyRatio} struct}, label={lst:polyratio-as-string}]
        pub fn as_string(&self) -> String {
            if self.denominator.as_string() == "1".to_string() {
                self.numerator.as_string()
            } else if self.denominator.as_string() == "0".to_string() {
                "ERROR: Division by zero!".to_string()
            } else {
                format!(
                    "({}) / ({})",
                    self.numerator.as_string(),
                    self.denominator.as_string()
                )
            }
        }
    \end{lstlisting}

    \begin{lstlisting}[caption={The implementation of the \texttt{evaluate()} method for the \texttt{PolyRatio} struct}, label={lst:polyratio-evaluate}]
        pub fn evaluate(&mut self, values: &Vec<(String, Rational64)>) {
            self.numerator.evaluate(values);
            self.denominator.evaluate(values);
        }
    \end{lstlisting}
% }

The four elementary arithmetic operations are defined for \verb|PolyRatio| in the following way:
\begin{itemize}
    \item Addition is implemented by multiplying the denominators and adding the numerators, then simplifying the result.
    \item Subtraction is implemented by multiplying the denominators and subtracting the numerators, then simplifying the result.
    \item Multiplication is implemented by multiplying the numerators and denominators separately, then simplifying the result.
    \item Division is implemented by multiplying the numerator by the denominator of the divisor and the denominator by the numerator of the divisor (cross-multiplication), then simplifying the result. 
\end{itemize}

% \multilinecomment{
    \begin{lstlisting}[caption={The implementation of the addition operation for the \texttt{PolyRatio} struct}, label={lst:polyratio-add}]
        impl Add for PolyRatio {
            type Output = Self;
        
            fn add(self, other: Self) -> Self {
                let mut result = PolyRatio {
                    numerator: self.numerator.clone() * other.denominator.clone()
                        + other.numerator.clone() * self.denominator.clone(),
                    denominator: self.denominator.clone() * other.denominator.clone(),
                };
                result.simplify();
                result
            }
        }
    \end{lstlisting}
    
    \begin{lstlisting}[caption={The implementation of the subtraction operation for the \texttt{PolyRatio} struct}, label={lst:polyratio-sub}]
        impl Sub for PolyRatio {
            type Output = Self;
        
            fn sub(self, other: Self) -> Self {
                let mut result = PolyRatio {
                    numerator: self.numerator.clone() * other.denominator.clone()
                        - other.numerator.clone() * self.denominator.clone(),
                    denominator: self.denominator.clone() * other.denominator.clone(),
                };
                result.simplify();
                result
            }
        }
    \end{lstlisting}

    \begin{lstlisting}[caption={The implementation of the multiplication operation for the \texttt{PolyRatio} struct}, label={lst:polyratio-mul}]
        impl Mul for PolyRatio {
            type Output = Self;
        
            fn mul(self, other: Self) -> Self {
                let mut result = PolyRatio {
                    numerator: self.numerator.clone() * other.numerator.clone(),
                    denominator: self.denominator.clone() * other.denominator.clone(),
                };
                result.simplify();
                result
            }
        }
    \end{lstlisting}

    \begin{lstlisting}[caption={The implementation of the division operation for the \texttt{PolyRatio} struct}, label={lst:polyratio-div}]
        impl Div for PolyRatio {
            type Output = Self;
        
            fn div(self, other: Self) -> Self {
                let mut result = PolyRatio {
                    numerator: self.numerator.clone() * other.denominator.clone(),
                    denominator: self.denominator.clone() * other.numerator.clone(),
                };
                result.simplify();
                result
            }
        }
    \end{lstlisting}
% }

A set of compatibility functions are also implemented for \verb|PolyRatio| to make it easier to work with \verb|Polynomial|s:

\begin{itemize}
    \item The \verb|from()| function converts a \verb|Polynomial| to a \verb|PolyRatio| by setting the denominator to $1$.
    \item The four elementary arithmetic operations are implemented between \verb|Polynomial| and \verb|PolyRatio|. These operations are implemented by upgrading the \verb|Polynomial| using \verb|from()| and performing the operation as defined for \verb|PolyRatio|. These are actually methods of the \verb|Polynomial| struct.
\end{itemize}

% \multilinecomment{
    \begin{lstlisting}[caption={The implementation of the \texttt{from()} function for the \texttt{PolyRatio} struct}, label={lst:polyratio-from}]
        impl From<Polynomial> for PolyRatio {
            fn from(p: Polynomial) -> Self {
                PolyRatio {
                    numerator: p,
                    denominator: Polynomial {
                        terms: vec![Term {
                            coefficient: Rational64::new(1, 1),
                            variables: vec![],
                        }],
                        degree: 1.into(),
                    },
                }
            }
        }
    \end{lstlisting}

    \begin{lstlisting}[caption={The implementation of the addition operation between \texttt{Polynomial} and \texttt{PolyRatio}}, label={lst:polyratio-polynomial-add}]
        impl Add<PolyRatio> for Polynomial {
            type Output = PolyRatio;
        
            fn add(self, other: PolyRatio) -> PolyRatio {
                let upgraded_self = PolyRatio::from(self);
                upgraded_self + other
            }
        }
    \end{lstlisting}

    \begin{lstlisting}[caption={The implementation of the subtraction operation between \texttt{Polynomial} and \texttt{PolyRatio}}, label={lst:polyratio-polynomial-sub}]
        impl Sub<PolyRatio> for Polynomial {
            type Output = PolyRatio;
        
            fn sub(self, other: PolyRatio) -> PolyRatio {
                let upgraded_self = PolyRatio::from(self);
                upgraded_self - other
            }
        }
    \end{lstlisting}

    \begin{lstlisting}[caption={The implementation of the multiplication operation between \texttt{Polynomial} and \texttt{PolyRatio}}, label={lst:polyratio-polynomial-mul}]
        impl Mul<PolyRatio> for Polynomial {
            type Output = PolyRatio;
        
            fn mul(self, other: PolyRatio) -> PolyRatio {
                let upgraded_self = PolyRatio::from(self);
                upgraded_self * other
            }
        }
    \end{lstlisting}

    \begin{lstlisting}[caption={The implementation of the division operation between \texttt{Polynomial} and \texttt{PolyRatio}}, label={lst:polyratio-polynomial-div}]
        impl Div<PolyRatio> for Polynomial {
            type Output = PolyRatio;
        
            fn div(self, other: PolyRatio) -> PolyRatio {
                let upgraded_self = PolyRatio::from(self);
                upgraded_self / other
            }
        }
    \end{lstlisting}
% }

\section{Main program}\label{sec:main-program}

The main program is responsible for parsing, evaluating and performing operations on the symbolic expressions expressed by the custom syntax defined in the grammar. Using the Pest parser, the program reads the input file and performs the operations.

\subsection{Helper functions}\label{subsec:helper-functions}

The main program has several helper functions that are used to parse the input file:

\begin{itemize}
    \item The \verb|variable_from_string()| function converts a string (e.g.: \verb|x^(2)|) to a \verb|Variable| struct. It splits the string by the caret symbol (\verb|^|) if present, then converts the first part to a string and the second part (which can be a ratio, a floating-point number or an integer) to a \verb|Rational64|, and returns a \verb|Variable| struct.
    \item The \verb|parse_polynomial()| function parses a polynomial from the result of matching the \verb|polynomial| rule in the Pest parser (e.g.: \verb|x^(2) + 3y - 1|). It iterates through the parts of the parsed expression, identifying terms and factors within terms, according to the inner rules of the definition of a polynomial according to the grammar. The terms are accumulated and used to construct a \verb|Polynomial| struct.
    \item The \verb|parse_assignment()| function parses an assignment from the result of matching the \verb|assignment| rule in the Pest parser (e.g.: \verb|x = 2|). It extracts the variable name and the value, and stores them in a tuple, later to be pushed to the vector holding the variable-value tuples.
    \item The \verb|parse_operation()| function parses an operation from the result of matching the \verb|operation| rule in the Pest parser (e.g.: \verb|(4x) * (y + 6) / (2)|) into a \verb|PolyRatio| struct. It parses the first polynomial in the operation sequence, and then iterates through the rest of the operation sequence, parsing the next polynomial, matching the operation with its corresponding rule and performing the calculation using the previous result.
\end{itemize}

% \multilinecomment{
    \begin{lstlisting}[caption={The implementation of the \texttt{variable\_from\_string()} function}, label={lst:variable-from-string}]
        fn variable_from_string(var: &str) -> polynomial::Variable {
            let mut iter = var.split('^');
            let name = iter.next().unwrap().to_string();
            let degree = iter
                .next()
                .map(|d| {
                    let clean = d.replace("(", "").replace(")", "");
                    if d.contains('/') {
                        let parts: Vec<&str> = clean.split('/').collect();
                        let numerator = parts[0].trim().parse::<i64>().unwrap_or(1);
                        let denominator = parts[1].trim().parse::<i64>().unwrap_or(1);
                        Rational64::new(numerator, denominator)
                    } else if d.contains('.') {
                        let parts: Vec<&str> = clean.split('.').collect();
                        let numerator = clean.replace(".", "").trim().parse::<i64>().unwrap_or(1);
                        let denominator = 10_i64.pow(parts[1].len() as u32);
                        Rational64::new(numerator, denominator)
                    } else {
                        clean.trim().parse::<i64>().unwrap_or(1).into()
                    }
                })
                .unwrap_or(1.into());
            polynomial::Variable { name, degree }
        }
    \end{lstlisting}
    
    \begin{lstlisting}[caption={The implementation of the \texttt{parse\_polynomial()} function}, label={lst:parse-polynomial}]
        fn parse_polynomial(expression: Pairs<Rule>) -> polynomial::Polynomial {
            let mut p = polynomial::Polynomial {
                terms: Vec::new(),
                degree: 1.into(),
            };
            for part in expression {
                match part.as_rule() {
                    Rule::term => {
                        let mut term = polynomial::Term {
                            coefficient: Rational64::new(1, 1),
                            variables: Vec::new(),
                        };
                        for factor in part.into_inner() {
                            match factor.as_rule() {
                                Rule::sign => {
                                    if factor.as_str() == "-" {
                                        term.coefficient *= -1;
                                    }
                                }
                                Rule::number => {
                                    term.coefficient *=
                                        factor.as_str().trim().parse::<Rational64>().unwrap();
                                }
                                Rule::fraction => {
                                    let mut iter = factor.into_inner();
                                    let numerator =
                                        iter.next().unwrap().as_str().trim().parse::<i64>().unwrap();
                                    let denominator =
                                        iter.next().unwrap().as_str().trim().parse::<i64>().unwrap();
                                    term.coefficient *= Rational64::new(numerator, denominator);
                                }
                                Rule::var => {
                                    let variable = variable_from_string(factor.as_str());
                                    term.variables.push(variable);
                                }
                                Rule::EOI => (),
                                _ => unreachable!(),
                            }
                        }
                        p.terms.push(term);
                    }
                    Rule::EOI => (),
                    _ => unreachable!(),
                }
            }
            p
        }
    \end{lstlisting}
    
    \begin{lstlisting}[caption={The implementation of the \texttt{parse\_assignment()} function}, label={lst:parse-assignment}]
        fn parse_assignment(assignment: Pairs<Rule>) -> (String, Rational64) {
            // Only assigning Rational64 values?
            let mut iter = assignment;
            let var_name = iter.next().unwrap().as_str().to_string();
            let var_value = iter
                .next()
                .unwrap()
                .as_str()
                .trim()
                .parse::<Rational64>()
                .unwrap();
            (var_name, var_value)
        }
    \end{lstlisting}
    
    \begin{lstlisting}[caption={The implementation of the \texttt{parse\_operation()} function}, label={lst:parse-operation}]
        fn parse_operation(operation: Pairs<Rule>) -> polynomial::PolyRatio {
            let mut iter = operation;
            let first_poly = parse_polynomial(iter.next().unwrap().into_inner());
            let mut result = PolyRatio::from(first_poly);
        
            while let Some(op) = iter.next() {
                let next_poly = parse_polynomial(iter.next().unwrap().into_inner());
                match op.as_rule() {
                    Rule::add => result = result + next_poly,
                    Rule::sub => result = result - next_poly,
                    Rule::mul => result = result * next_poly,
                    Rule::div => result = result / next_poly,
                    _ => unreachable!(),
                }
            }
            result
        }        
    \end{lstlisting}
% }

\subsection{Main function}\label{subsec:main-function}

The main function orchestrates the overall flow of the program by reading input, parsing it, and performing various operations based on the parsed data.

First, it reads the input file into a string. If the file cannot be read, the programm will panic and terminate with an error message indicating the issue.

If the file is read successfully, the program will parse the input using the Pest parser that was generated based on the custom grammar. If the input cannot be parsed, the program will panic with an error message indicating the issue.

After the two previous steps have been successfuly executed, we can initialize storage for the vector \verb|var_values|, where we will store the variable-value tuples with all the custom definition the user inputs.

\begin{lstlisting}[caption={Reading and parsing user input, and defining storage for custom variable values}, label={lst:reading-parsing-input}]
    let unparsed_file = fs::read_to_string("input.txt").unwrap();

    let file = PolyParser::parse(Rule::file, &unparsed_file)
    .expect("unsuccessful parse")
    .next()
    .unwrap();

    let mut var_values: Vec<(String, Rational64)> = Vec::new();
\end{lstlisting}

Now we can start parsing each of the lines in the input file. When a line is read, it will be skipped if it is empty. If not, the input line will be printed to let the user know what operation is being performed.

After that, each of the lines is matched to one of the rules defined in the grammar as possible \verb|expr|(ession) types. This includes a \verb|polynomial|, an \verb|assignment|, an \verb|operation|, or an equation to \verb|solve|.

Here is ano verview of the behavior when matching each of the rules:
\begin{itemize}
    \item \textbf{Assignment}:
    \begin{itemize}
        \item Call the \verb|parse_assignment()| function to extract the variable name and value.
        \item Add the variable-value tuple to the \verb|var_values| vector.
        \item Print the assignment to the user.
    \end{itemize}
    \item \textbf{Polynomial}:
    \begin{itemize}
        \item Call the \verb|parse_polynomial()| function to parse the polynomial in the line.
        \item Evaluate the polynomial using the stored variable values.
        \item Print the evaluation and simplification of polynomial.
    \end{itemize}
    \item \textbf{Operation}:
    \begin{itemize}
        \item Call the \verb|parse_operation()| function to parse the operation in the line.
        \item Evaluate the resulting \verb|PolyRatio| using the stored variable values.
        \item Print the evaluation and simplification of the result of the operation.
    \end{itemize}
    \item \textbf{Solve}:
    \begin{itemize}
        \item Call the \verb|parse_polynomial()| function to parse the equation in the line. It is assumed that the right-hand side of the equation is always $0$.
        \item Identify which variable is being solved for. The user may have specified it after the equation (e.g.: \texttt{[3x - 1, x]}). If not, then the first variable in the equation is assumed to be the one being solved for.
        \begin{itemize}
            \item In case no variable is found in the equation, the program will panic and terminate with an error message. 
        \end{itemize}
        \item Solve the equation for the variable by calling the \verb|solve()| method of the \verb|Polynomial| struct.
        \item Print the solution to the user, handling the case where there are multiple solutions.
    \end{itemize}
\end{itemize}

An additional rule exists for handling end of input (EOI). Any other unreachable cases that should not occur if the grammar is properly followed in the input file will cause the program to panic and terminate with an error message.

\begin{lstlisting}[caption={Iterating through the lines of the input file}, label={lst:iterating-input}]
    for line in file.into_inner() {
    if line.as_str().trim().is_empty() {
        continue; // Skip empty lines
    }
    println!("{}", line.as_str());
    match line.as_rule() {
        Rule::assign => {
            let (var_name, var_value) = parse_assignment(line.into_inner());
            var_values.push((var_name.clone(), var_value));

            println!("\t{} = {}", var_name, var_value);
        }
        Rule::polynomial => {
            let mut p = parse_polynomial(line.into_inner());
            p.evaluate(&var_values);
            println!("\t{}", p.as_string());
        }
        Rule::operation => {
            let mut result = parse_operation(line.into_inner());
            result.evaluate(&var_values);
            println!("\t{}", result.as_string());
        }
        Rule::solve => {
            let mut iter = line.into_inner();
            let mut p = parse_polynomial(iter.next().unwrap().into_inner());
            p.evaluate(&var_values);
            let mut variable = p.first_var().unwrap_or("".to_string());
            if variable.is_empty() {
                panic!("No variable to solve for");
            }
            if let Some(var) = iter.next() {
                variable = var.as_str().to_string();
            }
            let result = p.roots(&variable);
            for root in result {
                if root.len() == 1 {
                    println!("\t{}\t= {}", variable, root[0].as_string());
                } else if root.len() > 1 {
                    print!("\t{}\\t= {}", variable, root[0].as_string());
                    for ratio in &root[1..] {
                        if ratio.numerator.terms[0].coefficient >= 0.into() {
                            print!(" + ");
                        }
                        print!("{}", ratio.as_string());
                    }
                    println!();
                }
            }
        }
        Rule::EOI => (),
        _ => unreachable!(),
    }
}
\end{lstlisting}
